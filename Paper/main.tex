%
%

% -----------
% 1. Präambel
% -----------

% Allgemeine Einstellungen
% ------------------------
\documentclass[
    pdftex,
	a4paper,
	oneside,
	12pt,
	liststotocnumbered
]{article}

% Meta Information festlegen
\PassOptionsToPackage{hyphens}{url}\usepackage[
	pdftitle={},
	pdfsubject={},
	pdfauthor={},
	pdfkeywords={},
	colorlinks=false,
	breaklinks=true
]{hyperref}

\usepackage{times}            % Times New Roman
\usepackage{newclude}
\usepackage[ampersand]{easylist}

\usepackage{tabularx}
\usepackage{footnote}
\usepackage[utf8]{inputenc}   % utf-8
\usepackage[T1]{fontenc}      % Umlauttrennung
\usepackage[english]{babel}    % englische Silbentrennung
\selectlanguage{english}       % englische Betitelung
\usepackage{nicefrac}

% Titel-Font-Größen
\usepackage{titlesec}
\titleformat{\section}{\bfseries}{\thesection.}{12pt}{}
\titleformat{\subsection}{\bfseries}{\thesubsection }{12pt}{}

% Seitenränder
\usepackage[
    top=2.5cm, 
    bottom=2.5cm, 
    left=5cm, 
    right=1cm
]{geometry} 

% Fussnoten
% multiple
\usepackage[hang,flushmargin]{footmisc}    
\renewcommand*{\footnotelayout}{\footnotesize} % size of text
\renewcommand{\footnotemargin}{2.2em}          % margin between text and number
\setlength{\footnotesep}{1.3em}                % space between footnotes
\setlength{\skip\footins}{2.5em}               % space between text & footnotes
\usepackage{savefnmark}

% Mehrere Fussnoten mit Komma separieren
\let\oldFootnote\footnote
\newcommand\nextToken\relax

\renewcommand\footnote[1]{%
    \oldFootnote{#1}\futurelet\nextToken\isFootnote}

\newcommand\isFootnote{%
    \ifx\footnote\nextToken\textsuperscript{, }\fi}

% Abkürzungen
%[printonlyused]
\usepackage{acronym}
\renewcommand{\bflabel}[1]{{#1\hfill}}

% Seitennummerierung oben
\usepackage{scrpage2} 
\usepackage[dvipsnames,usenames]{color}
\clearscrheadfoot 
\chead[\pagemark]{\textcolor[gray]{0.5}{\pagemark}} 
\pagestyle{scrheadings}

% TOC, LOF, FIG Styles
\usepackage{needspace}
\usepackage{tocloft, titletoc}  
\setlength{\cftaftertoctitleskip}{0em}
\renewcommand{\cftloftitlefont}{\bfseries}
%\renewcommand{\cftfigfont}{\bfseries}
\renewcommand{\cfttoctitlefont}{\bfseries}
\renewcommand{\cftlottitlefont}{\bfseries}
\titlecontents{section}     % set formatting for \section 
[2.3em]                     % adjust left margin
{\vspace{0.5em}}            % font formatting
{\hspace{-1.8em}.\contentslabel{0.7em}\hspace{1em}} % section label and offset
{\hspace*{-2.3em}}
{\titlerule*[1mm]{.}\contentspage}

\titlecontents{subsection}  % set formatting for \subsection 
[3em]                       % adjust left margin
{\vspace{0.5em}}            % font formatting
{\contentslabel{2.3em}}     % section label and offset
{\hspace*{-2.3em}}
{\titlerule*[1mm]{.}\contentspage}

\titlecontents{subsubsection}  % set formatting for \subsubsection 
[4.2em]                       % adjust left margin
{\vspace{0.5em}}            % font formatting
{\contentslabel{2.3em}}     % section label and offset
{\hspace*{-2.3em}}
{\titlerule*[1mm]{.}\contentspage}

\titlecontents{figure}      % set formatting for \subsection 
[3.4em]                     % adjust left margin
{\vspace{0.5em}}            % font formatting
{:\hspace*{0.9em}\contentslabel{4.5em}}     % section label and offset
{\hspace*{-2.3em}}
{\titlerule*[1mm]{.}\contentspage}

\titlecontents{table}       % set formatting for \subsection 
[3.4em]                     % adjust left margin
{\vspace{0.5em}}            % font formatting
{:\hspace*{0.9em}\contentslabel{4.5em}}     % section label and offset
{\hspace*{-2.3em}}
{\titlerule*[1mm]{.}\contentspage}


% Literaturverzeichnis
\usepackage[
    bibstyle=authortitle,
    citestyle=authoryear,
    isbn=false,
    url=true,
    doi=false,
    maxcitenames=3,
    maxbibnames=30
]{biblatex}
\addbibresource{literature.bib}
\let\cite\textcite

\usepackage{caption}
\usepackage{chngcntr}

% Tabellenpackete
\usepackage{array}
\usepackage{xcolor}
\usepackage{longtable}
\usepackage{setspace}
\counterwithin{table}{section}
\usepackage{multirow}
\usepackage{longtable}
\usepackage{float}
\restylefloat{table}
\usepackage{enumitem}
\setenumerate{noitemsep,topsep=0pt,parsep=0pt,partopsep=0pt}

% Grafiken anzeigen
\usepackage[pdftex]{graphicx}
\graphicspath{{figures}}
\counterwithin{figure}{section}
\usepackage[absolute,overlay]{textpos}

\setlength{\parindent}{0in}
\interfootnotelinepenalty=10000

\begin{document}
%\shorthandoff{"}

% Variables

% Renews
\renewcommand{\figurename}{}
\renewcommand{\tablename}{}
\renewcommand\thefigure{Fig. \arabic{section}-\arabic{figure}}
\renewcommand\thetable{Tab. \arabic{section}-\arabic{table}}
\newcommand{\todo}[1]{\textbf{\textsc{\textcolor{Red}{TODO: #1}}}}
\newcommand{\note}[1]{\textbf{\textsc{\textcolor{Cyan}{NOTE: #1}}}}
\renewcommand{\contentsname}{Table of Contents}
\renewcommand{\listtablename}{Index of Tables}
\renewcommand{\listfigurename}{Index of Illustrations}
\renewcommand{\arraystretch}{1.5}

\setcounter{secnumdepth}{5}

\pagenumbering{Roman}

% ---------
% Deckblatt
% ---------
\vspace*{1mm}

% Name
\thispagestyle{empty}
Sven van de Eynden, Manuel Schmidt-Kraeplin, Phil Diegmann

\vspace*{32mm}

% 
\begin{center}
\textbf{
    Selected Issues II
\linebreak
    Major Information Systems
}
\end{center}

\vspace*{32mm}

% Titel
\begin{center}
\LARGE 
Benefits of Augmented Reality in Educational Environments
\end{center}

\vspace*{32mm}

% Themensteller
\begin{center}
Dr. Dirk Basten
\end{center}

\vspace*{32mm}

% Köln, ...
\begin{center}
Köln, Juni 2014
\end{center}

% ------------------
% Inhaltsverzeichnis
% ------------------
\tocloftpagestyle{scrheadings}
\tableofcontents
\newpage

% ---------------------
% Abkürzungsverzeichnis
% ---------------------
\section*{Index of Abbreviations}
\addcontentsline{toc}{section}{Index of Abbreviations}
\begin{longtable}{@{}p{.275\textwidth}@{}p{.725\textwidth}@{}}
\end{longtable}

% -------------------
% Tabellenverzeichnis
% -------------------
\listoftables
\addcontentsline{toc}{section}{Index of Tables}
\newpage

% -------------------
% Abbildungsverzeichnis
% -------------------
\listoffigures
\addcontentsline{toc}{section}{Index of Illustrations}
\newpage

\normalsize
\setstretch{1,5}
\pagenumbering{arabic}

% ------
% Inhalt
% ---------

\renewcommand{\arraystretch}{1.0}

% --------------------
% Literaturverzeichnis
% --------------------

\newpage
\patchcmd{\bibsetup}{\interlinepenalty=5000}{\interlinepenalty=10000}{}{}
\printbibliography[title={Bibliography}]
\addcontentsline{toc}{section}{Bibliography}


% ---------
% Anhang
% ---------

\end{document}