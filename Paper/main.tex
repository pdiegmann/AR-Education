%
%

% -----------
% 1. Präambel
% -----------

% Allgemeine Einstellungen
% ------------------------
\documentclass[
    pdftex,
	a4paper,
	oneside,
	12pt,
	liststotocnumbered
]{article}

% Meta Information festlegen
\PassOptionsToPackage{hyphens}{url}\usepackage[
	pdftitle={Benefits of Augmented Reality in Educational Environments},
	pdfsubject={},
	pdfauthor={Sven van de Eynden, Manuel Schmidt-Kraeplin, Phil Diegmann},
	pdfkeywords={Augmented Reality, Education, AR, Benefits, Educational},
	colorlinks=false,
	breaklinks=true
]{hyperref}

\usepackage{times}            % Times New Roman
\usepackage{newclude}
\usepackage[ampersand]{easylist}

\usepackage{tabularx}
\usepackage{footnote}
\usepackage{csquotes}
\usepackage[utf8]{inputenc}   % utf-8
\usepackage[T1]{fontenc}      % Umlauttrennung
\usepackage[english]{babel}    % englische Silbentrennung
\selectlanguage{english}       % englische Betitelung
\usepackage{nicefrac}

% Titel-Font-Größen
\usepackage{titlesec}
\titleformat{\section}{\bfseries}{\thesection.}{12pt}{}
\titleformat{\subsection}{\bfseries}{\thesubsection }{12pt}{}

% Seitenränder
\usepackage[
    top=2.5cm, 
    bottom=2.5cm, 
    left=5cm, 
    right=1cm
]{geometry} 

% Fussnoten
% multiple
\usepackage[hang,flushmargin]{footmisc}    
\renewcommand*{\footnotelayout}{\footnotesize} % size of text
\renewcommand{\footnotemargin}{2.2em}          % margin between text and number
\setlength{\footnotesep}{1.3em}                % space between footnotes
\setlength{\skip\footins}{2.5em}               % space between text & footnotes
\usepackage{savefnmark}

\newlength{\arrayrulewidthOriginal}
\newcommand{\Cline}[2]{%
  \noalign{\global\setlength{\arrayrulewidthOriginal}{\arrayrulewidth}}%
  \noalign{\global\setlength{\arrayrulewidth}{#1}}\cline{#2}%
  \noalign{\global\setlength{\arrayrulewidth}{\arrayrulewidthOriginal}}}

% Mehrere Fussnoten mit Komma separieren
\let\oldFootnote\footnote
\newcommand\nextToken\relax

\renewcommand\footnote[1]{%
    \oldFootnote{#1}\futurelet\nextToken\isFootnote}

\newcommand\isFootnote{%
    \ifx\footnote\nextToken\textsuperscript{, }\fi}

% Abkürzungen
%[printonlyused]
\usepackage{acronym}
\renewcommand{\bflabel}[1]{{#1\hfill}}

% Kurzwahlen Directions
\newcommand{\AR}{Augmented Reality }
\newcommand{\DBL}{Discovery-based Learning }
\newcommand{\OM}{Objects Modelling }
\newcommand{\ARB}{AR Books }
\newcommand{\ST}{Skills Training }
\newcommand{\ARG}{AR Gaming }
% ns = no space
\newcommand{\ARns}{Augmented Reality}
\newcommand{\DBLns}{Discovery-based Learning}
\newcommand{\OMns}{Objects Modelling}
\newcommand{\ARBns}{AR Books}
\newcommand{\STns}{Skills Training}
\newcommand{\ARGns}{AR Gaming}

% superscript comma and space
\newcommand{\mulcit}{\textsuperscript{, }}

% Seitennummerierung oben
\usepackage{scrpage2} 
\usepackage[dvipsnames,usenames]{color}
\clearscrheadfoot 
\chead[\pagemark]{\textcolor[gray]{0.5}{\pagemark}} 
\pagestyle{scrheadings}

% TOC, LOF, FIG Styles
\usepackage{needspace}
\usepackage{tocloft, titletoc}  
\setlength{\cftaftertoctitleskip}{0em}
\renewcommand{\cftloftitlefont}{\bfseries}
%\renewcommand{\cftfigfont}{\bfseries}
\renewcommand{\cfttoctitlefont}{\bfseries}
\renewcommand{\cftlottitlefont}{\bfseries}
\titlecontents{section}     % set formatting for \section 
[2.3em]                     % adjust left margin
{\vspace{0.5em}}            % font formatting
{\hspace{-1.8em}.\contentslabel{0.7em}\hspace{1em}} % section label and offset
{\hspace*{-2.3em}}
{\titlerule*[1mm]{.}\contentspage}

\titlecontents{subsection}  % set formatting for \subsection 
[3em]                       % adjust left margin
{\vspace{0.5em}}            % font formatting
{\contentslabel{2.3em}}     % section label and offset
{\hspace*{-2.3em}}
{\titlerule*[1mm]{.}\contentspage}

\titlecontents{subsubsection}  % set formatting for \subsubsection 
[4.2em]                       % adjust left margin
{\vspace{0.5em}}            % font formatting
{\contentslabel{2.3em}}     % section label and offset
{\hspace*{-2.3em}}
{\titlerule*[1mm]{.}\contentspage}

\titlecontents{figure}      % set formatting for \subsection 
[3.4em]                     % adjust left margin
{\vspace{0.5em}}            % font formatting
{:\hspace*{0.9em}\contentslabel{4.5em}}     % section label and offset
{\hspace*{-2.3em}}
{\titlerule*[1mm]{.}\contentspage}

\titlecontents{table}       % set formatting for \subsection 
[3.4em]                     % adjust left margin
{\vspace{0.5em}}            % font formatting
{:\hspace*{0.9em}\contentslabel{4.5em}}     % section label and offset
{\hspace*{-2.3em}}
{\titlerule*[1mm]{.}\contentspage}


% Literaturverzeichnis
\usepackage[
    bibstyle=authortitle,
    citestyle=authoryear-ibid,
    isbn=false,
    url=true,
    doi=false,
    maxcitenames=2,
    maxbibnames=30,
    autocite=footnote
]{biblatex}
\addbibresource{literature.bib}
\let\cite\textcite

\usepackage{caption}
\usepackage{chngcntr}

% Tabellenpackete
\usepackage{array}
\usepackage{xcolor}
\usepackage{longtable}
\usepackage{setspace}
\counterwithin{table}{section}
\usepackage{multirow}
\usepackage{longtable}
\usepackage{float}
\restylefloat{table}
\usepackage{enumitem}
\usepackage{changepage}
\usepackage{pdflscape}
\setenumerate{noitemsep,topsep=0pt,parsep=0pt,partopsep=0pt}

% Grafiken anzeigen
\usepackage[pdftex]{graphicx}
\graphicspath{{figures}}
\usepackage{rotating}
\counterwithin{figure}{section}
\usepackage[absolute,overlay]{textpos}

\setlength{\parindent}{0in}
\interfootnotelinepenalty=10000

\begin{document}

%\shorthandoff{"}

% Variables

% Renews
\renewbibmacro*{cite}{%
  \global\boolfalse{cbx:loccit}%
  \iffieldundef{shorthand}
    {\ifthenelse{\ifciteibid\AND\NOT\iffirstonpage}
       {\usebibmacro{cite:ibid}}
       {\ifthenelse{\ifnameundef{labelname}\OR\iffieldundef{labelyear}}
          {\usebibmacro{cite:label}%
           \setunit{\addspace}}
          {\printnames{labelname}%
           \setunit{\nameyeardelim}}%
        \iffieldundef{labelyear}
          {}
          {\printtext[parens]{\usebibmacro{cite:labelyear+extrayear}}}}}
    {\usebibmacro{cite:shorthand}}}
\renewcommand{\figurename}{}
\renewcommand{\tablename}{}
\renewcommand\thefigure{Fig. \arabic{section}-\arabic{figure}}
\renewcommand\thetable{Tab. \arabic{section}-\arabic{table}}
\newcommand{\todo}[1]{\textbf{\textsc{\textcolor{Red}{TODO: #1}}}}
\newcommand{\note}[1]{\textbf{\textsc{\textcolor{Cyan}{NOTE: #1}}}}
\renewcommand{\contentsname}{Table of Contents}
\renewcommand{\listtablename}{Index of Tables}
\renewcommand{\listfigurename}{Index of Illustrations}
\renewcommand{\arraystretch}{1.5}

\setcounter{secnumdepth}{5}

\pagenumbering{Roman}

% ---------
% Deckblatt
% ---------
\vspace*{1mm}

% Name
\thispagestyle{empty}
Phil Diegmann, Manuel Schmidt-Kraeplin, Sven van de Eynden

\vspace*{32mm}

% 
\begin{center}
\textbf{
    Selected Issues II
\linebreak
    Major Information Systems
}
\end{center}

\vspace*{32mm}

% Titel
\begin{center}
\LARGE 
Benefits of Augmented Reality in Educational Environments
\end{center}

\vspace*{32mm}

% Themensteller
\begin{center}
Dr. Dirk Basten
\end{center}

\vspace*{32mm}

% Köln, ...
\begin{center}
Köln, Juni 2014
\end{center}

% ------------------
% Inhaltsverzeichnis
% ------------------
\tocloftpagestyle{scrheadings}
\tableofcontents
\newpage

% ---------------------
% Abkürzungsverzeichnis
% ---------------------
\section*{Index of Abbreviations}
\addcontentsline{toc}{section}{Index of Abbreviations}
\begin{longtable}{@{}p{.275\textwidth}@{}p{.725\textwidth}@{}}
\end{longtable}

% -------------------
% Tabellenverzeichnis
% -------------------
\listoftables
\addcontentsline{toc}{section}{Index of Tables}
\newpage

% -------------------
% Abbildungsverzeichnis
% -------------------
\listoffigures
\addcontentsline{toc}{section}{Index of Illustrations}
\newpage

\normalsize
\setstretch{1,5}
\pagenumbering{arabic}

% ------
% Inhalt
% ---------
\section{Introduction}
\subsection{Problem Statement}
I cite.\footnote{\cite{Chang.2014}\label{Chang_2014}}\\
And again.\footnote{\cite{Andujar.2011}} Or again the first footnote.\footref{Chang_2014}
\subsection{Objectives}
\subsection{Definition of "Augmented Reality"}
\subsection{Augmented Reality in Educational Environments}
\section{Augmented Reality in Educational Environments}

\subsection{Definition of "Augmented Reality"}
Although the term 'Augmented Reality' was coined by Tom Caudell, a former Boeing researcher, in 1990, the concept of augmenting the real world with virtual data was initially used by a number of applications in the late 1960s and 1970s. Since the 1990s, AR was used by some large companies in purpose of visualization and training. Nowerdays, the rising power of personal computers and mobile devices enable the concept of AR to be delivered to traditional educational environments such as schools and universities. \autocite [cf.][21]{Johnson.2010} 
\begin{figure}[ptbh]
    \centering
    \includegraphics[width=\linewidth]{figures/rvc.png}
    \caption[Reality-Virtuality Continuum]{Reality-Virtuality Continuum}
    \label{fig:RealityVirtualityContinuum}
\end{figure}

During the last years the term 'Augmented Reality' has been given different meanings by varying researchers. \autocite [cf.][42]{Wu.2013} \cite{Milgram.1994b} defined AR on the basis of the reality-virtuality continuum (\ref{fig:RealityVirtualityContinuum}) as "augmenting natural feedback to the operator with simulated cues". \autocite[283]{Milgram.1994b} The reality-virtuality continuum (\ref{fig:RealityVirtualityContinuum}) allows us to distinguish the concept of AR to related concepts such as Virtual Reality (VR) where "the participant observer is totally immersed in a completely synthetic world" \autocite[283]{Milgram.1994b} or Augmented Virtuality (AV) where "the the primary world being experienced is in fact [...] predominantly 'virtual'"\autocite[4]{Milgram.1994} and augmented with information from the real world. In addition, \cite{Milgram.1994b} mention a more restricted definiton where AR is seen as "form of virtual reality where the participant's head-mounted display is transparent, allowing a clear view of the real world". \autocite[283]{Milgram.1994b} As suggested by educational researchers,\autocite[cf.][42]{Wu.2013} we reject the idea that the concept of AR is limited to any type of technology. Therefore, we broadly define AR referring to \cite{Klopfer.2008} as "a situation in which a real world context is dynamically overlaid with coherent location or context sensitive virtual information"\autocite[205]{Klopfer.2008} and regard it as a concept which is based on and realized by but conzeptualized beyond technology.

\subsection{Five Directions of Augmented Reality in Educational Environments}
There are several different ways how the concept of AR is implemented in educational environments. \autocite {Yuen.2011}\mulcit\autocite {Lee.2012} The Five Directions by \cite{Yuen.2011} enable us to classify the AR applications we are investigating in our systematic literature review into five groups which are introduced in the following. This classification helps us to find out how the benefits of AR in educational environments differ dependent on the regarded type of AR application.
\heading{Discovery-based Learning}
AR is often used in applications that enable discovery-based learning. Therefore, the user is provided with information about a real-world place while simultaneously regarding the object of interest. This type of application is often used in museums, astronomical education or at historical places.
\heading{Objects Modeling}
AR is also used in objects modeling applications. These applications allow students to receive immediate visual feedback of how a given item would look like in a different setting. Some applications also allow students to design virtual objects in order to investigate their physical properties or interactions between objects. This type of application is also used in architectural education.
\heading{AR Books}
AR books are books which offer students 3D presentations and interactive learning experiences through AR technology. The books are augmented with the help of technological devices such as special glasses. The first implementations of AR books show, that this kind of medium is likely to appeal to digital native learners which makes it an appropriate educational medium even at the primary level.
\heading{Skills Training}
The support of training individuals in specific tasks is described by the term 'Skills Training'. AR skills training applications are used for example in airplane maintenance, where each step of a repair is displayed, necessary tools are identified and textual instructions are included. Skills training applications are often realised with head-mounted displays. 
\heading{AR Gaming}
Video Games offer powerful new opportunities for educators which have been ignored for many years. \autocite{Squire.2003} Nowerdays, educators have recognized and often use the power of games and gamification in educational environments. AR technology enables the development of games which take place in the real world and are augmented with virtual information. This type of games can give educators powerful new ways to show relationships and connections. In addition, AR games provide educators with highly interactive and visual forms of learning.
\section{Systematic Literature Review}
We applied a two-step research approach, whereby we first conducted a systematic literature review to identify relevant publications before analysing the identified publications for the coding of benefits and directions. After coding, we grouped all found benefits. This process is illustrated in \ref{fig:ResearchApproachGathering} for data collection and in \ref{fig:ResearchApproachAnalysis} for data analysis.

\subsection{Data Collection}
For the identification of papers addressing Augmented Reality in educational environments, we applied a systematic online literature database search. We included databases which were specialised on more information systems centered papers, namely Institute of Electrical and Electronic Engineers (IEEE) Xplore Digital Library, ProQuest (ABI / INFORM), Association for Information Systems Electronic Library (AISel) and Association for Computing Machinery (ACM) Digital Library, as well as more general databases, namely EBSCO Host and ScienceDirect.\\
To find relevant papers, we searched within all databases with on the following attributes: title, abstract and author supplied keywords. Within these keywords we had three mandatory groups of keywords. Every article had to include the keyword "Augmented Reality". Additionally, every article needed to have at least one synonym for education and benefits. Namely we searched for "Educat*", "Learn*", "Teach*", "College" or "School" as synonyms for education and "Benefi*", "Advan*", "Improv*", "Enhanc*", "Driver*" or "Value*" as synonyms for benefits. To deal with the limitations of some databases, we had to split our query and conduct multiple queries on the database and merge them together by hand.\\
This database query resulted in a total of 523 articles. Those results were checked against our include- and exclude-criteria, which are listed in \ref{tab:IncludeExcludeCriteria}, and were preliminary coded into one of the five directions. This process was performed by ourselves and each article was read by two of the authors.\\
After merging our results, a total of 25 articles remained.

\begin{figure}[ptbh]
    \centering
    \includegraphics[width=\linewidth]{figures/research_approach_part_1.png}
    \caption[Research Approach: Data Gathering]{Research Approach: Data Gathering}
    \label{fig:ResearchApproachGathering}
\end{figure}

\begin{figure}[ptbh]
    \centering
    \includegraphics[width=\linewidth]{figures/research_approach_part_2.png}
    \caption[Research Approach: Data Analysis]{Research Approach: Data Analysis}
    \label{fig:ResearchApproachAnalysis}
\end{figure}

\begin{table}[ptbh]
    \center
    \vspace{1em}
    \begin{tabular}{p{17em} | p{17em}}
        \textbf{Include Criteria} & \textbf{Exclude Criteria} \\
        \hline
        Empirical works & Theoretical works, grey literature, dissertations \\
        A teaching problem is solved with the help of Augmented Reality or a teaching concept is improved by Augmented Reality & Untried or untested technologies, concepts without empirical evidence \\
        Lists positive effects of Augmented Reality applications in comparison to conventional learning tools & No control-group or control-scenario provided, no comparison to conventional learning tools \\
        Human learning & Machine learning \\
        English language & Other language \\
        Peer-reviewed & Not peer-reviewed \\
    \end{tabular}
    \caption[Include- and Exclude-Criteria]{Include- and Exclude-Criteria}
    \label{tab:IncludeExcludeCriteria}
\end{table}

\subsection{Data Analysis}
During data analysis we clustered all found benefits into clusters and matched all found benefits to the directions of the articles in which they were mentioned. We will go into details regarding the benefits found and the grouping of them in chapter \ref{subsec:Benefits} and regarding the mapping of benefits to directions in chapter \ref{subsec:Mapping}.\\
Because of our orientation towards the five directions, we assigned directions to all articles during data collection but revised our assignment in case of differences between the first and second coder. Our inter coder reliability score is 0.64. We will interpret and discuss this score in chapter \ref{sec:Discussion}. During assignment of directions we also collected all mentioned benefits and generalised similar benefits into a single one. Afterwards, those benefits were grouped into broader topic-related benefits. The process we applied is based on the process proposed by Jankowicz.\autocite[cf.][149]{Jankowicz.2004} The process proposed by Jankowicz helps by formalising the process of clustering.\\
A total of 67 benefits were mentioned, containing 14 unique benefits, which were clustered into six clusters.
\section{Benefits of Augmented Reality in Educational Environments}
\subsection{Benefit Categorization}
\label{subsec:Benefits}

% 
\subsubsection{State of Mind}
In this subsection all benefits are presented which we grouped together under the terms 'State of Mind'. These benefits are related to the users state of mind while using the \AR \app. Some of these benefits can affect 
each other, such as increased motivation and increased attention. If there is an attention for the \AR \app this could be higher, the greater the motivation of the user is. Indeed the 
benefits differ in certain properties, as you can see in the following. \\
\heading{Increased Motivation}
By increased motivation we mean that users had more eager and are more engaged to deal with the new technology and thus also to deal with the teaching and learning content than by application of non-\AR methods. With a fraction 21,74\% of all benefits mentioned, 
'increased motivation' is after 'improved learning curve' (26.87\%) by far the most mentioned benefit. This is shown by some simple quotations by \cite{Iwata.2011}, \cite{Yen.2013}, \cite{Ke Tian.2013} and
\cite[et.al.][]{Kamarainen.2013}, who present this benefit literally, such as "[...] the AR-style game play successfully enhanced intrinsic motivation towards the self-learning process"\autocite[113]{Iwata.2011}, "Participants 
using the AR books appeared much more eager at the beginning of each session compared with the NAR group"\autocite[et.al.][112]{Andreas.2012} or "Students have been satisfied and motivated by these new methodologies, in all cases"
\autocite[et.al.][60]{Redondo.2013}. Furthermore it is also shown by some implicit statements like "Results showed that students were less bored and more in flow state
when the AR-based application was used during the Magnet_2 stage"\autocite[et.al.][8]{Ibanez.2014} or by findings such as more proactivity\autocite[et.al.][10]{Chang.2014} \autocite or the will to continue learning using
the AR-Technology after class\autocite[8]{Liu.2009}.
\heading{Increased Attention}
\heading{Increased Concentration}
\heading{Increased Satisfaction}

% 
\subsubsection{Teaching Concepts}
During our analysis we observed that two different teaching concepts were supported by AR applications. We clustered these concepts as 'Student Centered Learning' and 'Collaborative Learning' which we explain in the following.
\heading{Increased Student Centered Learning}
Student centered learning is a teaching concept where conventional lectures are replaced by new active and self-paced learning programs. In student centered learning approaches, students are more self-responsible for their own progress in education and educators act as faciliators, who enable the students to learn independently and individualized.\\
Three studies report that AR enabled an increased student centered learning approach in the regarded learning environment. \cite{VateULan.2012} recognizes that the regarded AR application enabled "functionality depended on [...] students’ learning capability" \autocite [894]{VateULan.2012}. Similarly, \cite{Kamarainen.2013} report that "these technologies provide ways of individualizing instruction in a group setting".\autocite[554]{Kamarainen.2013} In addition, \cite{Kamarainen.2013} state that "the technology supported independence" which "freed the teacher to act as a faciliator".\autocite[554]{Kamarainen.2013} These studies show, that AR can support a student centered learning approach by providing educators with new possibilites to individualize their lessons to students' capability and by enabling students to learn more independently from educators.
\heading{Improved Collaborative Learning}
Three studies report that the regarded AR application improved collaborative learning, meaning that AR enabled new ways of communication and cooperation. \cite{Wang.2012} regard their AR application as "effective environment for conducting collaborative inquiry learning activities". \autocite[57]{Wang.2012} Other authors join the observation of improved collaborative learning as they highlight "the opportunity for collaborative communication and problem-solving among students that arose from the augmented reality experience" \autocite[552]{Kamarainen.2013} and the "facilitation effects of AR technology on collaborative learning effectiveness".\autocite[322]{Li.2011}
% 
\subsubsection{Presentation}
\heading{Increased Details}
\heading{Increased Information Accessibility}
\heading{Increased Interactivity}

% 
\subsubsection{Learning Type}
This subsection deals with benefits we clustered as 'Learning Type'. This group contains benefits which were linked to a specific type of learning, for instance creativity or a more theoretical learning approach like language education. \\
Therefor this group contains two subitems: improved learning curve and increased creativity. While an improved learning curve is observable on skills based learning, such as spatial skills, or on fields which require a logical understanding, such as languages, increased creativity can be observed on less theoretical grounded areas, such as problem solving or arts.

\heading{Improved Learning Curve}
An improved learning curve, meaning that students learn faster and easier with \AR \apps compared to non-\AR \appsns, is the most often mentioned benefit of \ARns. A total of 26.87\% of all benefits mentioned were related to an improved learning curve. \\
\cite{Liu.2009} reports that "tests taken by the experimental group [the \AR \app users] in all the learning activities were significantly better than those of the control group [the traditionally learning users]".\autocite[525]{Liu.2009} Similarly, \cite{Chang.2014} states, that "[t]he AR-guided group had better learning effectiveness (as evidenced by their posttest scores), and it was found that most visitors believed the AR guide made it easier to digest information than the audio guide due to the extra visual commentary that is provided"\autocite[193]{Chang.2014} as well as "[t]he learning performance of the AR-guided group was thus superior to
that of the other two groups"\autocite[190]{Chang.2014}. More authors join this observation like \cite{Kamarainen.2013} ("[w]e witnessed significant learning gains"\autocite[550]{Kamarainen.2013}), \cite{Ibanez.2014} ("it was found that students who used the AR application performed significantly better on knowledge"\autocite[12]{Ibanez.2014}), \cite{Li.2011}, \cite{MartinGutierrez.2011}, \cite{Redondo.2013}, \cite{Liu.2009b} ("achieved significantly more learning improvement"\autocite[173]{Liu.2009b}), \cite{Zhang.2014}, \cite{Yeo.2011}, \cite{Hou.2013} ("[AR] shortens the learning curve"\autocite[450]{Hou.2013}, "[the] learning curve of trainees significantly improved"\autocite[451]{Hou.2013}), \cite{Wilson.2013} and \cite{Anderson.2013} ("learning [results] increased by more than a factor of 2"\autocite[318]{Anderson.2013}). 
\heading{Increased Creativity}
Increased creativity was mentioned three times (which makes 4,48\% of all reported benefits). For instance, \cite{Liu.2009b} found that "it [AR] also improves student creativity and the ability to explore and absorb new knowledge and solve problems"\autocite[173]{Liu.2009b}. \cite{VateULan.2012} reports, that the "AR 3D pop-up book has highlighted many benefits that include: [...] integration of a variety of learning skills such as [...] and creativity [...]"\autocite[894]{VateULan.2012}. Also, \cite{Chang.2014} observes, that "[o]verall the visitors using the mobile AR-guide system during painting appreciation activities felt that it was an interesting, innovative, creative, and entertaining guide device"\autocite[194]{Chang.2014}. To increase the interpretability of the impact of \AR \apps on creativity, more studies are needed.
% 
\subsubsection{Content Understanding}
\heading{Improved Development of Spacial Abilities}
\heading{Improved Memory}

% 
\subsubsection{Reduced Cost}
\cite{Leblanc.2010} and \cite{MartinGutierrez.2011} reported reduced costs in \ARns-scenarios compared to traditional learning in long term. \cite{Chen.2012} highlights especially the low cost in executing manpower and moderate costs for designing and renewing of courses.\autocite[cf.][640]{Chen.2012} \cite{Andujar.2011} join in this point, especially for virtual laboratories.\autocite[cf.][492]{Andujar.2011} \cite{Andujar.2011} add that \ARns-\apps not only reduce direct costs, such as needed materials, but also time for preparing classes. While, at least at the time of this review, \ARns-technology is accompanied with high aquisition cost, this investment will most likely be paid off in the long term. \cite{Leblanc.2010} report, that the one time acquisition cost were high (25.000 US-Dollar)\autocite[253]{Leblanc.2010}, but the cost per class could be lowered by 93,34\% (from 3.000 US-Dollar to 200 US-Dollar)\autocite[253]{Leblanc.2010} which will lead to an overall cost reduction.
\subsection{Mapping of the Benefits to the "Five Directions"}
\label{subsec:Mapping}
Following, we will present the mapping of the found benefits to the five directions. In \ref{tab:MapBenefitsDirections} the mapping results are listed in detail. \\
As highlighted in \ref{subsec:DataAnalysis}, we followed the theoretical approach of clustering proposed by \cite{Jankowicz.2004}.\autocite[cf.][149]{Jankowicz.2004} First, we assigned articles to one of the Five Directions by \cite{Yuen.2011}.\autocite[cf.][127-130]{Yuen.2011} The definitions by \cite{Yuen.2011} state different aspects and characteristics for every direction, which we tried to match to the reviewed articles. After the assignment of a direction to each article, we counted the occurrences of each benefit found in the articles for each direction. Our results will be presented below.

\subsubsection{\DBLns}
\label{subsubsec:DiscoveryBasedLearning}
We found eight articles (32.00\% of all articles in our result set) which presented learning concepts were Discovery-based. Those articles had the most mentions of state of mind benefits, especially increased motivation. 47.00\% of all increased motivation benefits were related to a Discovery-based \AR \appns. Also, an improved learning curve was mentioned. About one third of all improved learning curves were observed in \DBL environments. Nine out of 14 benefits were reported for \DBL \apps (64.29\%), which is the most diverse pool of benefits we found during our literature review. Reduced costs were reported in one article for \DBL \appsns.

\subsubsection{\OMns}
In our result set of 25 articles, we found five articles (20.00\% of all articles reviewed), which dealt with an \OM approach for the presented \AR \appns. Similar to \DBL \appsns, \OM resulted in an increased motivation and satisfaction. We found about 26.67\% of all mentions of increased motivation in an \OM context. Also, an improved learning curve was observed. About 22.22\% of all mentions of an improved learning curve were in coherence with an \OM \appns. It is noticeable, that although \OM itself is highly interactive, we did not find any references of an increased interactivity in classes which used \AR than in classes which did not. None of the \OM \apps mention presentation-linked benefits. Also, we found no reports of increased creativity linked to \OM, but spatial abilities were reported to be developed better. Five different benefits were found in \OM \appsns, which is about 35.71\% of all reported unique benefits. \OM \apps are reported to have reduced costs in comparison to non-\AR learning tools.

\subsubsection{\ARBns}
Two articles (which makes a total of 8.00\%) were found which were based on an \ARB \appns. \ARB \apps were the least found direction in the reviewed articles. \ARB \apps are also connected to an increase in motivation, but not as much as \DBL or \OMns. \ARB seem to provide balanced benefits. Six of 14 benefits were reported in context of \ARB which makes about 42.86\%. No reduced costs were reported for \ARB \apps.

\subsubsection{\STns}
We found seven articles (28.00\% of all articles) which presented a \ST \AR \appns. 50.00\% (seven out of 14) of all unique benefits were also mentioned in \ST \appsns. \ST \apps have the most mentions of content understanding, especially in improved memory. It is furthermore worth noticing, that \ST \apps have the same count of mentions for improved learning curves as \DBL \appsns. Both have the highest count for improved learning curves. It was reported that \ST \apps reduced the costs in comparison to traditional learning tools.

\subsubsection{\ARGns}
\ARG was presented in three articles of our result set which accounts for 12.00\%. \ARG has most benefits in the state of mind group. An improved learning curve as well as better accessible information were reported. Content understanding and teaching concepts, such as collaborative learning, were not explicitly improved in the reviewed cases. Reduced costs were reported for \ARG \apps from one article.

\begin{landscape}
\begin{table}[!htb]
    \center
    \vspace{1.5cm}
    \resizebox{0.77\textwidth}{!}{\begin{minipage}{\textwidth}
    \begin{adjustwidth}{-6.5cm}{}
    \begin{tabular}{c c || c | c | c | c | c || c}
        \textbf{} & \textbf{} & \textbf{\DBLns} & \textbf{\OMns} & \textbf{\ARBns} & \textbf{\STns} & \textbf{\ARGns} & Sums \\
        %\hline
        \Cline{1.0pt}{1-8}
        \textbf{State of Mind} & Increased Motivation & 7 & 4 & 2 & 1 & 1 & 15 \\
        \cline{2-8}
        & Increased Attention & 2 & 0 & 1 & 0 & 0 & 3 \\
        \cline{2-8}
        & Increased Concentration & 2 & 0 & 0 & 0 & 1 & 3 \\
        \cline{2-8}
        & Increased Satisfaction & 1 & 2 & 0 & 1 & 1 & 5 \\
         \cline{2-8}
        % & Sums & 12 & 6 & 3 & 2 & 3 & \\
        \Cline{1.0pt}{1-8}
        \textbf{Teaching} & Student Centered & 2 & 0 & 1 & 0 & 0 & 3 \\ \textbf{Concepts} & Learning & & & & & \\
        \cline{2-8}
        & Improved Collective & 1 & 2 & 0 & 0 & 0 & 3 \\ & Learning & & & & & \\
         \cline{2-8}
        % & Sums & 3 & 2 & 1 & 0 & 0 & \\
        \Cline{1.0pt}{1-8}
        \textbf{Presentation} & Increased Details & 0 & 0 & 0 & 1 & 0 & 1 \\
        \cline{2-8}
        & Easy Accessible & 0 & 0 & 0 & 1 & 1 & 2 \\ & Information & & & & & \\
        \cline{2-8}
        & Interactivity & 1 & 0 & 1 & 0 & 0 & 2 \\
         \cline{2-8}
        % & Sums & 1 & 0 & 1 & 2 & 1 & \\
        \Cline{1.0pt}{1-8}
        \textbf{Learning} & Improved Learning & 6 & 4 & 1 & 6 & 1 & 18 \\ \textbf{Types} & Curve & & & & & \\
        \cline{2-8}
        & Increased Creativity & 2 & 0 & 1 & 0 & 0 & 3 \\
         \cline{2-8}
        % & Sums & 8 & 4 & 2 & 6 & 1 & \\
        \Cline{1.0pt}{1-8}
        \textbf{Reduced Costs} & Reduced Costs & 0 & 1 & 0 & 1 & 0 & 2 \\
        \Cline{1.0pt}{1-8}
        \textbf{Content} & Development of & 0 & 2 & 1 & 1 & 0 & 4 \\ \textbf{Understanding} & Spatial Abilities & & & & & \\
        \cline{2-8}
        & Improved Memory & 1 & 0 & 0 & 2 & 0 & 3 \\
        \cline{2-8}
        % & Sums & 1 & 2 & 1 & 3 & 0 & \\
    \end{tabular}
    
    \end{adjustwidth}
    \begin{adjustwidth}{-8.5cm}{}
    \caption[Mapping of Benefits and Directions]{Mapping of Benefits and Directions (25 articles, six benefit groups, 14 different benefits and five directions) }
    \label{tab:MapBenefitsDirections}
    \end{adjustwidth}
    \end{minipage}}
\end{table}
\end{landscape}
\section{Discussion}
\label{sec:Discussion}
Hier blabla discussion \\

Our study is limited by a number of factors. Firstly, some of the regarded empirical studies are only informal investigations with a low number of participants. The significance of the ascertained benefits of AR applications may be unclear in these cases. In addition, for some of the regarded directions we did not find enough articles in order to make a point about the diversity of benefits in comparison to other directions. However, AR is one of the most emerging technologies in education and the fact that 15 out of 25 articles we regarded were published in 2012 or later shows that these limitations can be overcome in the future when more empirical evaluations of AR applications in educational environments will be published. Once enough articles have been published we would suggest to investigate every direction of AR in education separately with a decent amount of regarded articles in order to find out more about the diversity of benefits between directions. Another factor which limits our study is revealed by the inter-code reliability of 0.64 regarding the classification of articles to a certain direction of AR. We think that this rather low value can be explained by the circumstance that some articles can not precisely be classified to a single direction, e.g. a discovery-based learning application which uses game elements. In addition, the definitions by \cite{Yuen.2011} leave some room for interpretation which we tried to reduce during our systematic literature review.
\section{Conclusion}
%Evtl Zusätzlich in limitations (oder Conclusion): Jede AR application ist einzigartig, daher sind benefits nur sehr schwer generalisierbar. Eine Applikation muss auf jeden Fall vernünftig umgesetzt sein (z.B. ausreichendes Maß an usability) um von den potentiellen Vorteilen zu profitieren.
%Evtl Zusätzlich in limitations (oder conclusion): Wir haben keine "Special Learner" betrachtet. Gerade für diese Gruppe ist aber AR besonders interessant! (Evtl. Zitat) Erklären warum das notwendig war und dass man diese Gruppen gesondert betrachten sollte, da hier erhebliches Potential für education besteht!

\renewcommand{\arraystretch}{1.0}

% --------------------
% Literaturverzeichnis
% --------------------

\newpage
\patchcmd{\bibsetup}{\interlinepenalty=5000}{\interlinepenalty=10000}{}{}
\printbibliography[title={Bibliography}]
\addcontentsline{toc}{section}{Bibliography}


% ---------
% Anhang
% ---------

\end{document}