\section{Introduction}
\subsection{Problem Statement}
Bridging the gap between virtual and real worlds, Augmented Reality (AR) provides new possibilities of teaching and learning which have been increasingly recognized by educational reaserchers. \autocite [cf.][41]{Wu.2013} Although AR is one of the most emerging technologies in education in these days \autocite [cf.]{Johnson.2010}, the unique value of AR learning environments remains unclear. \autocite [cf.][48]{Wu.2013} In addition, there are different types of AR applications in educational environments which may differ regarding their benefits towards educational outcomes. \autocite [cf.][127-130]{Yuen.2011} This leads to the problem that teachers and professors may be not aware of potential benefits of AR applications in comparison to conventional learning tools. \\
This work considers two main research questions: 1.Which benefits are provided by an AR application in comparison to conventional learning tools? 2. How do those benefits differ regarding the different types of AR applications in educational environments? \\
Although recent studies have investigated the use of AR in educational environments \autocite {Wu.2013}\mulcit\autocite {Lee.2012} and a first approach to consolidate AR Benefits in educational environments has been made, \autocite [cf.]{Radu.2014} more evidence on the educational values of AR is needed. Unfortunately, we were not able to replicate the approach by Radu(2014) in order to search for additional benefits because of missing information towards the applied methodology. In addition this approach does not consider the different types of AR applications in educational environments.\\
An overview of the benefits of AR in educational environments regarding the different types of AR applications would help teachers and professors to decide whether the implementation of AR is reasonable in certain educational scenarios.\\


\subsection{Objectives}
The main objective of this work is to identify benefits which can be provided by AR as a learning medium in educational environments in comparison to conventional learning tools. Besides, we want find out how these benefits differ regarding the different types of AR in educational environments. \\
To fulfill these objectives we conducted a systematic literature review to identify and analyze relevant publications. In addition, we clustered relevant publications with regard to the applied type of AR on the basis of the 'Five Directions of AR in education' by Yuen et al. (2011)
\subsection{Structure}