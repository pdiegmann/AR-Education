\section{Introduction}
\subsection{Problem Statement}
Praxisproblem: The value of Augmented Reality applications in educational environments is not obvious in many cases[Zitat]. In addition, there are different types of augmented reality applications in educational environments which may differ regarding their benefits towards educational outcomes[Five Directions Zitat]. This leads to the fact that teachers and professor may be not aware of potential benefits of augmented reality applications in comparison to conventional learning tools.[Zitat suchen]\\
Relevanz: The use of AR in educational environments such as US Colleges has been increased during the last few years [Horizon Report]. This shows, that AR has become an important factor in supporting educational environments.\\
Forschungsfrage: Which Benefits are provided by an AR application in comparison to conventional learning tools? \\
Forschungsproblem: A first approach to consolidate AR Benefits in educational environments has been made. Unfortunately, the approach by Radu can not be replicated because of missing information towards the regarded databases and include / exclude criteria. In addition this approach does not consider the different types of AR applications in educational environments and we think that there might be additional benefits which are not mentioned by Radu.\\
Relevanz: An overview of the benefits of AR in educational environments regarding the different types of AR applications can help teachers and professors to decide whether the implementation of AR is reasonable in certain educational scenarios.\\
Forschungsantwort: Overview of the Benefits of AR in educational environments and mapping of different types of AR applications and AR Benefits.

I cite.\autocite[cf.][149]{Chang.2014}\\
And again.\autocite[4-5]{Dunser.2012}
\subsection{Objectives}
\subsection{Structure}