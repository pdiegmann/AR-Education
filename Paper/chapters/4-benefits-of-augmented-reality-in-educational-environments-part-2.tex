\subsection{Mapping of the Benefits to the „Five Directions“}
\label{subsec:Mapping}
Following, we will present the mapping of the found benefits to the five directions. In \ref{tab:MapBenefitsDirections} the mapping results are listed in detail. \\
As highlighted in \ref{subsec:DataAnalysis}, we followed the theoretical approach of clustering proposed by \cite{Jankowicz.2004}.\autocite[cf.][149]{Jankowicz.2004} First, we assigned articles to one of the Five Directions by \cite{Yuen.2011}.\autocite[cf.][127-130]{Yuen.2011} The definitions by \cite{Yuen.2011} state different aspects and characteristics for every direction, which we tried to match to the reviewed articles. After the assignment of a direction to each article, we counted the occurrences of each benefit found in the articles for each direction. Our results will be presented below.

\subsubsection{\DBLns}
We found eight articles (32.00\% of all articles in our result set) which presented learning concepts were Discovery-based. Those articles had the most mentions of state of mind benefits, especially increased motivation. 47.00\% of all increased motivation benefits were related to a Discovery-based \AR \appns. Also, an improved learning curve was mentioned. About one third of all improved learning curves were observed in \DBL environments. Nine out of 14 benefits were reported for \DBL \apps (64.29\%), which is the most diverse pool of benefits we found during our literature review. Reduced costs were reported in one article for \DBL \appsns.

\subsubsection{\OMns}
In our result set of 25 articles, we found five articles (20.00\% of all articles reviewed), which dealt with an \OM approach for the presented \AR \appns. Similar to \DBL \appsns, \OM resulted in an increased motivation and satisfaction. We found about 26.67\% of all mentions of increased motivation in an \OM context. Also, an improved learning curve was observed. About 22.22\% of all mentions of an improved learning curve were in coherence with an \OM \appns. It is noticeable, that although \OM itself is highly interactive, we did not find any references of an increased interactivity in classes which used \AR than in classes which did not. None of the \OM \apps mention presentation-linked benefits. Also, we found no reports of increased creativity linked to \OM, but spatial abilities were reported to be developed better. Five different benefits were found in \OM \appsns, which is about 35.71\% of all reported unique benefits. \OM \apps are reported to have reduced costs in comparison to non-\AR learning tools.

\subsubsection{\ARBns}
Two articles (which makes a total of 8.00\%) were found which were based on an \ARB \appns. \ARB \apps were the least found direction in the reviewed articles. \ARB \apps are also connected to an increase in motivation, but not as much as \DBL or \OMns. \ARB seem to provide balanced benefits. Six of 14 benefits were reported in context of \ARB which makes about 42.86\%. No reduced costs were reported for \ARB \apps.

\subsubsection{\STns}
We found seven articles (28.00\% of all articles) which presented a \ST \AR \appns. 50.00\% (seven out of 14) of all unique benefits were also mentioned in \ST \appsns. \ST \apps have the most mentions of content understanding, especially in improved memory. It is furthermore worth noticing, that \ST \apps have the same count of mentions for improved learning curves as \DBL \appsns. Both have the highest count for improved learning curves. It was reported that \ST \apps reduced the costs in comparison to traditional learning tools.

\subsubsection{\ARGns}
\ARG was presented in three articles of our result set which accounts for 12.00\%. \ARG has most benefits in the state of mind group. An improved learning curve as well as better accessible information were reported. Content understanding and teaching concepts, such as collaborative learning, were not explicitly improved in the reviewed cases. Reduced costs were reported for \ARG \apps from one article.

\begin{landscape}
\begin{table}[!htb]
    \center
    \vspace{1.5cm}
    \resizebox{0.77\textwidth}{!}{\begin{minipage}{\textwidth}
    \begin{adjustwidth}{-6.5cm}{}
    \begin{tabular}{c c || c | c | c | c | c || c}
        \textbf{} & \textbf{} & \textbf{\DBLns} & \textbf{\OMns} & \textbf{\ARBns} & \textbf{\STns} & \textbf{\ARGns} & Sums \\
        %\hline
        \Cline{1.0pt}{1-8}
        \textbf{State of Mind} & Increased Motivation & 7 & 4 & 2 & 1 & 1 & 15 \\
        \cline{2-8}
        & Increased Attention & 2 & 0 & 1 & 0 & 0 & 3 \\
        \cline{2-8}
        & Increased Concentration & 2 & 0 & 0 & 0 & 1 & 3 \\
        \cline{2-8}
        & Increased Satisfaction & 1 & 2 & 0 & 1 & 1 & 5 \\
         \cline{2-8}
        % & Sums & 12 & 6 & 3 & 2 & 3 & \\
        \Cline{1.0pt}{1-8}
        \textbf{Teaching} & Student Centered & 2 & 0 & 1 & 0 & 0 & 3 \\ \textbf{Concepts} & Learning & & & & & \\
        \cline{2-8}
        & Improved Collective & 1 & 2 & 0 & 0 & 0 & 3 \\ & Learning & & & & & \\
         \cline{2-8}
        % & Sums & 3 & 2 & 1 & 0 & 0 & \\
        \Cline{1.0pt}{1-8}
        \textbf{Presentation} & Increased Details & 0 & 0 & 0 & 1 & 0 & 1 \\
        \cline{2-8}
        & Easy Accessible & 0 & 0 & 0 & 1 & 1 & 2 \\ & Information & & & & & \\
        \cline{2-8}
        & Interactivity & 1 & 0 & 1 & 0 & 0 & 2 \\
         \cline{2-8}
        % & Sums & 1 & 0 & 1 & 2 & 1 & \\
        \Cline{1.0pt}{1-8}
        \textbf{Learning} & Improved Learning & 6 & 4 & 1 & 6 & 1 & 18 \\ \textbf{Types} & Curve & & & & & \\
        \cline{2-8}
        & Increased Creativity & 2 & 0 & 1 & 0 & 0 & 3 \\
         \cline{2-8}
        % & Sums & 8 & 4 & 2 & 6 & 1 & \\
        \Cline{1.0pt}{1-8}
        \textbf{Reduced Costs} & Reduced Costs & 0 & 1 & 0 & 1 & 0 & 2 \\
        \Cline{1.0pt}{1-8}
        \textbf{Content} & Development of & 0 & 2 & 1 & 1 & 0 & 4 \\ \textbf{Understanding} & Spatial Abilities & & & & & \\
        \cline{2-8}
        & Improved Memory & 1 & 0 & 0 & 2 & 0 & 3 \\
        \cline{2-8}
        % & Sums & 1 & 2 & 1 & 3 & 0 & \\
    \end{tabular}
    
    \end{adjustwidth}
    \begin{adjustwidth}{-8.5cm}{}
    \caption[Mapping of Benefits and Directions]{Mapping of Benefits and Directions (25 articles, six benefit groups, 14 different benefits and five directions) }
    \label{tab:MapBenefitsDirections}
    \end{adjustwidth}
    \end{minipage}}
\end{table}
\end{landscape}