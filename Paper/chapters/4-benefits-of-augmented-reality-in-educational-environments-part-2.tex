\subsection{Mapping of the Benefits to the „Five Directions“}
\label{subsec:Mapping}
Following, we will present the mapping of the found benefits to the five directions. In \ref{tab:MapBenefitsDirections} the mapping results are listed in detail. \\
\textbf{TODO: Describe pattern and procedure? Maybe more Detailed in 3.2?}

\subsubsection{\DBLns}
We found eight articles (32.00\% of all articles in our result set) whose presented learning concepts were Discovery-based. Those articles had the most mentions of State of Mind benefits, especially increased motivation. 47\% of all increased motivation benefits were related to a Discovery-based \AR application. Also, an improved learning curve was mentioned. About one-third of all improved learning curves were observed in \DBL environments. Nine out of 14 benefits were reported for \DBL applications (about 64.29\%), which is the most diverse pool of benefits we found during our literature review. No reduced costs were reported for \DBL applications.

\subsubsection{\OMns}
In our result set of 25 articles, we found five articles (20.00\% of all articles reviewed), which dealt with an \OM approach for the presented \AR application. Similar to \DBL applications, \OM resulted in an increased motivation and satisfaction. We found about 26.67\% of all mentions of increased motivation in an \OM context. Also, an improved learning curve was observed. About 22.22\% of all mentions of an improved learning curve were coherence with an \OM application. It is noticeable, that although \OM itself is highly interactive, we did not find any references of an increased interactivity in classes which used \AR than in classes which did not. None of the \OM applications mention presentation-linked benefits. Also, we found no reports of increased creativity linked to \OM, but spatial abilities were reported to be developed better. Five different benefits were found in \OM applications, which is about 35.71\% of all reported unique benefits. Besides \ST applications, \OM applications were the only ones which were reported to have reduced costs in comparison to non-\AR learning tools.

\subsubsection{\ARBns}
Two articles (which makes a total of 8.00\%) were found which were based on an \ARB application. \ARB applications were the least found in the articles. \ARB applications are also connected to an increase in motivation, but not as much as \DBL or \OMns. \ARB seem to provide balanced benefits. Six of 14 benefits were reported in context of \ARB which makes about 42.86\%. No reduced costs were reported for \ARB applications.

\subsubsection{\STns}
We found seven articles (28.00 \% of all articles) which presented a \ST \AR application. 50.00\% (seven out of 14) of all unique benefits were also mentioned in \ST applications. \ST applications have the most mentions of content understanding, especially in improved memory. It is furthermore worth noticing, that \ST applications have the same count of mentions for improved learning curves as \DBL applications. Both have the highest count for improved learning curves. Like \OM applications, it was reported, that \ST applications reduced the costs in comparison to traditional learning tools.

\subsubsection{\ARGns}
\ARG was presented in three articles of our result set which accounts for 12.00 \%. No reduced costs were reported for \ARG applications.

\begin{landscape}
\begin{table}[!htb]
    \center
    \begin{adjustwidth}{-1.25cm}{}
    \vspace{-4.45cm}
    \begin{tabular}{c c || c | c | c | c | c || c}
        \textbf{} & \textbf{} & \textbf{\DBLns} & \textbf{\OMns} & \textbf{\ARBns} & \textbf{\STns} & \textbf{\ARGns} & Sums \\
        %\hline
        \Cline{1.0pt}{1-8}
        \textbf{State of Mind} & Increased Motivation & 7 & 4 & 2 & 1 & 1 & 15 \\
        \cline{2-8}
        & Increased Attention & 2 & 0 & 1 & 0 & 0 & 3 \\
        \cline{2-8}
        & Increased Concentration & 2 & 0 & 0 & 0 & 1 & 3 \\
        \cline{2-8}
        & Increased Satisfaction & 1 & 2 & 0 & 1 & 1 & 5 \\
         \cline{2-8}
         & Sums & 12 & 6 & 3 & 2 & 3 & \\
        \Cline{1.0pt}{1-8}
        \textbf{Teaching} & Student Centered & 2 & 0 & 1 & 0 & 0 & 3 \\ \textbf{Concepts} & Learning & & & & & \\
        \cline{2-8}
        & Improved Collective & 1 & 2 & 0 & 0 & 0 & 3 \\ & Learning & & & & & \\
         \cline{2-8}
         & Sums & 3 & 2 & 1 & 0 & 0 & \\
        \Cline{1.0pt}{1-8}
        \textbf{Presentation} & Increased Details & 0 & 0 & 0 & 1 & 0 & 1 \\
        \cline{2-8}
        & Easy Accessible & 0 & 0 & 0 & 1 & 1 & 2 \\ & Information & & & & & \\
        \cline{2-8}
        & Interactivity & 1 & 0 & 1 & 0 & 0 & 2 \\
         \cline{2-8}
         & Sums & 1 & 0 & 1 & 2 & 1 & \\
        \Cline{1.0pt}{1-8}
        \textbf{Learning} & Improved Learning & 6 & 4 & 1 & 6 & 1 & 18 \\ \textbf{Types} & Curve & & & & & \\
        \cline{2-8}
        & Increased Creativity & 2 & 0 & 1 & 0 & 0 & 3 \\
         \cline{2-8}
         & Sums & 8 & 4 & 2 & 6 & 1 & \\
        \Cline{1.0pt}{1-8}
        \textbf{Reduced Costs} & Reduced Costs & 0 & 1 & 0 & 1 & 0 & 2 \\
        \Cline{1.0pt}{1-8}
        \textbf{Content} & Development of & 0 & 2 & 1 & 1 & 0 & 4 \\ \textbf{Understanding} & Spatial Abilities & & & & & \\
        \cline{2-8}
        & Improved Memory & 1 & 0 & 0 & 2 & 0 & 3 \\
        \cline{2-8}
         & Sums & 1 & 2 & 1 & 3 & 0 & \\
    \end{tabular}
    \end{adjustwidth}
    \caption[Mapping of Benefits and Directions]{Mapping of Benefits and Directions}
    \label{tab:MapBenefitsDirections}
\end{table}
\end{landscape}