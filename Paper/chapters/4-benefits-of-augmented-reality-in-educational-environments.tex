\section{Benefits of Augmented Reality in Educational Environments}
\subsection{Benefit Categorization}
\label{subsec:Benefits}
\subsubsection{State of Mind}
\paragraph*{Increased Motivation}
\paragraph*{Increased Attention}
\paragraph*{Increased Concentration}
\paragraph*{Increased Satisfaction}
\subsubsection{Teaching Concepts}
\paragraph*{Increased Student Centered Learning}
\paragraph*{Improved Collaborative Learning}
\subsubsection{Presentation}
\paragraph*{Increased Details}
\paragraph*{Increased Information Accessibility}
\paragraph*{Increased Interactivity}
\subsubsection{Learning Type}
This subsection deals with benefits we clustered as 'Learning Type'. This group contains benefits which could be linked to a specific type of learning, for instance creativity or a more theoretical learning approach like language education. \\
Therefor this group contains two subitems: improved learning curve and improved creativity. While an improved learning curve is observable on skills based learning, such as spatial skills, or on fields which require a logical understanding, such as languages, improved creativity can be observed on less theoretical grounded areas, such as problem solving or arts.
\paragraph*{Improved Learning Curve}

\paragraph*{Increased Creativity}
\subsubsection{Content Understanding}
\paragraph*{Improved Development of Spacial Abilities}
\paragraph*{Improved Memory}
\subsubsection{Reduced Cost}