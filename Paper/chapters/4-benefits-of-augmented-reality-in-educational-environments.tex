\section{Benefits of Augmented Reality in Educational Environments}
\subsection{Benefit Categorization}
\label{subsec:Benefits}

% 
\subsubsection{State of Mind}
\heading{Increased Motivation}
\heading{Increased Attention}
\heading{Increased Concentration}
\heading{Increased Satisfaction}

% 
\subsubsection{Teaching Concepts}
\heading{Increased Student Centered Learning}
\heading{Improved Collaborative Learning}

% 
\subsubsection{Presentation}
\heading{Increased Details}
\heading{Increased Information Accessibility}
\heading{Increased Interactivity}

% 
\subsubsection{Learning Type}
This subsection deals with benefits we clustered as 'Learning Type'. This group contains benefits which could be linked to a specific type of learning, for instance creativity or a more theoretical learning approach like language education. \\
Therefor this group contains two subitems: improved learning curve and increased creativity. While an improved learning curve is observable on skills based learning, such as spatial skills, or on fields which require a logical understanding, such as languages, increased creativity can be observed on less theoretical grounded areas, such as problem solving or arts.

\heading{Improved Learning Curve}
An improved learning curve, meaning that students learn faster and easier with \AR \apps compared to non-\AR \appsns, is the most often mentioned benefit of \ARns. A total of 26.87\% of all benefits mentioned were related to an improved learning curve. 
\heading{Increased Creativity}
Increased creativity was observed in three articles (which is 4,48\% of all reviewed articles).
% 
\subsubsection{Content Understanding}
\heading{Improved Development of Spacial Abilities}
\heading{Improved Memory}

% 
\subsubsection{Reduced Cost}