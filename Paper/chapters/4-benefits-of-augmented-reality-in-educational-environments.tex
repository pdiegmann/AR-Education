\section{Benefits of Augmented Reality in Educational Environments}
\subsection{Benefit Categorization}
\label{subsec:Benefits}

% 
\subsubsection{State of Mind}
\heading{Increased Motivation}
\heading{Increased Attention}
\heading{Increased Concentration}
\heading{Increased Satisfaction}

% 
\subsubsection{Teaching Concepts}
During our analysis we observed that two different teaching concepts were supported by AR applications. We clustered these concepts as 'Student Centered Learning' and 'Collaborative Learning' which we explain in the following.
\heading{Increased Student Centered Learning}
Student centered learning is a teaching concept where conventional lectures are replaced by new active and self-paced learning programs. In student centered learning approaches, students are more self-responsible for their own progress in education and educators act as faciliators, who enable the students to learn independently and individualized.\\
Three studies report that AR enabled an increased student centered learning approach in the regarded learning environment. \cite{VateULan.2012} recognizes that the regarded AR application enabled "functionality depended on [...] students’ learning capability" \autocite [894]{VateULan.2012}. Similarly, \cite{Kamarainen.2013} report that "these technologies provide ways of individualizing instruction in a group setting".\autocite[554]{Kamarainen.2013} In addition, \cite{Kamarainen.2013} state that "the technology supported independence" which "freed the teacher to act as a faciliator".\autocite[554]{Kamarainen.2013} These studies show, that AR can support a student centered learning approach by providing educators with new possibilites to individualize their lessons to students' capability and by enabling students to learn more independently from educators.
\heading{Improved Collaborative Learning}
Three studies report that the regarded AR application improved collaborative learning, meaning that AR enabled new ways of communication and cooperation. \cite{Wang.2012} regard their AR application as "effective environment for conducting collaborative inquiry learning activities". \autocite[57]{Wang.2012} Other authors join the observation of improved collaborative learning as they highlight "the opportunity for collaborative communication and problem-solving among students that arose from the augmented reality experience" \autocite[552]{Kamarainen.2013} and the "facilitation effects of AR technology on collaborative learning effectiveness".\autocite[322]{Li.2011}
% 
\subsubsection{Presentation}
\heading{Increased Details}
\heading{Increased Information Accessibility}
\heading{Increased Interactivity}

% 
\subsubsection{Learning Type}
This subsection deals with benefits we clustered as 'Learning Type'. This group contains benefits which were linked to a specific type of learning, for instance creativity or a more theoretical learning approach like language education. \\
Therefor this group contains two subitems: improved learning curve and increased creativity. While an improved learning curve is observable on skills based learning, such as spatial skills, or on fields which require a logical understanding, such as languages, increased creativity can be observed on less theoretical grounded areas, such as problem solving or arts.

\heading{Improved Learning Curve}
An improved learning curve, meaning that students learn faster and easier with \AR \apps compared to non-\AR \appsns, is the most often mentioned benefit of \ARns. A total of 26.87\% of all benefits mentioned were related to an improved learning curve. \\
\cite{Liu.2009} reports that "tests taken by the experimental group [the \AR \app users] in all the learning activities were significantly better than those of the control group [the traditionally learning users]".\autocite[525]{Liu.2009} Similarly, \cite{Chang.2014} states, that "[t]he AR-guided group had better learning effectiveness (as evidenced by their posttest scores), and it was found that most visitors believed the AR guide made it easier to digest information than the audio guide due to the extra visual commentary that is provided"\autocite[193]{Chang.2014} as well as "[t]he learning performance of the AR-guided group was thus superior to
that of the other two groups"\autocite[190]{Chang.2014}. More authors join this observation like \cite{Kamarainen.2013} ("[w]e witnessed significant learning gains"\autocite[550]{Kamarainen.2013}), \cite{Ibanez.2014} ("it was found that students who used the AR application performed significantly better on knowledge"\autocite[12]{Ibanez.2014}), \cite{Li.2011}, \cite{MartinGutierrez.2011}, \cite{Redondo.2013}, \cite{Liu.2009b} ("achieved significantly more learning improvement"\autocite[173]{Liu.2009b}), \cite{Zhang.2014}, \cite{Yeo.2011}, \cite{Hou.2013} ("[AR] shortens the learning curve"\autocite[450]{Hou.2013}, "[the] learning curve of trainees significantly improved"\autocite[451]{Hou.2013}), \cite{Wilson.2013} and \cite{Anderson.2013} ("learning [results] increased by more than a factor of 2"\autocite[318]{Anderson.2013}). 
\heading{Increased Creativity}
Increased creativity was mentioned three times (which makes 4,48\% of all reported benefits). For instance, \cite{Liu.2009b} found that "it [AR] also improves student creativity and the ability to explore and absorb new knowledge and solve problems"\autocite[173]{Liu.2009b}. \cite{VateULan.2012} reports, that the "AR 3D pop-up book has highlighted many benefits that include: [...] integration of a variety of learning skills such as [...] and creativity [...]"\autocite[894]{VateULan.2012}. Also, \cite{Chang.2014} observes, that "[o]verall the visitors using the mobile AR-guide system during painting appreciation activities felt that it was an interesting, innovative, creative, and entertaining guide device"\autocite[194]{Chang.2014}. To increase the interpretability of the impact of \AR \apps on creativity, more studies are needed.
% 
\subsubsection{Content Understanding}
\heading{Improved Development of Spacial Abilities}
\heading{Improved Memory}

% 
\subsubsection{Reduced Cost}
\cite{Leblanc.2010} and \cite{MartinGutierrez.2011} reported reduced costs in \ARns-scenarios compared to traditional learning in long term. \cite{Chen.2012} highlights especially the low cost in executing manpower and moderate costs for designing and renewing of courses.\autocite[cf.][640]{Chen.2012} \cite{Andujar.2011} join in this point, especially for virtual laboratories.\autocite[cf.][492]{Andujar.2011} \cite{Andujar.2011} add that \ARns-\apps not only reduce direct costs, such as needed materials, but also time for preparing classes. While, at least at the time of this review, \ARns-technology is accompanied with high aquisition cost, this investment will most likely be paid off in the long term. \cite{Leblanc.2010} report, that the one time acquisition cost were high (25.000 US-Dollar)\autocite[253]{Leblanc.2010}, but the cost per class could be lowered by 93,34\% (from 3.000 US-Dollar to 200 US-Dollar)\autocite[253]{Leblanc.2010} which will lead to an overall cost reduction.