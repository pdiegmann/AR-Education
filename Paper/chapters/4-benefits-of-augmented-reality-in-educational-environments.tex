\section{Benefits of Augmented Reality in Educational Environments}
In this main-chapter we will present all benefits found. In the first sub-chapter we will present the benefits categorized in five different categories and afterwards we present the mapping of the benefit to the "Five Directions".
\subsection{Benefit Categorization}
\label{subsec:Benefits}
To improve clarity and to find semantically coherent groups, the benefits were clustered into a category if they are logically related to the same subject, as you can see in the following.

% 
\subsubsection{State of Mind}
In this subsection all benefits are presented which we grouped under the terms 'State of Mind'. These benefits are related to the users state of mind while using the \AR \appns. Some of these benefits can affect 
each other or are intuitive not clearly distinguished such as increased attention and increased concentration. Therefore some quotations could be interpreted as proof for another benefit as we did, as you can see in the following.
However the benefits differ in certain properties which we will point out in the respecting paragraphs.
\heading{Increased Motivation}
By increased motivation we mean that users have more eager and interest and are more engaged to deal with the new technology and thus also to deal with the teaching and learning content than by application of non-\AR methods. With a fraction 21,74\% of all benefits mentioned, 
'increased motivation' is after 'improved learning curve' (26.87\%) by far the most mentioned benefit, the third most are 'Reduced Cost' and 'Improved Development of Spacial Abilities' with fractions of 5,8\%.
This is shown by some quotations by \cite{Dunser.2012}, \cite{Iwata.2011}, 
\cite{Kamarainen.2013}, \cite{Liu.2009b}, \cite{MartinGutierrez.2011}, \cite{MartinGutierrez.2011}, \cite{Redondo.2013}, \cite{VateULan.2012} and \cite{Yen.2013}, who present this benefit literally, such as "[...] the AR-style game play successfully enhanced intrinsic motivation towards the self-learning process"\autocite[113]{Iwata.2011}, "Participants 
using the AR books appeared much more eager at the beginning of each session compared with the NAR group"\autocite[112]{Dunser.2012} or "students have been satisfied and motivated by these new methodologies, in all cases"
\autocite[60]{Redondo.2013}. Furthermore it is also shown by some implicit statements like "results showed that students were less bored and more in flow state
when the AR-based application was used during the Magnet\_2 stage"\autocite[8]{Ibanez.2014} or by findings such as the users "were more proactive"\autocite[10]{Chang.2014}\mulcit\autocite[cf.][187]{Zhang.2014} or the will to continue learning using
the AR-Technology after class\autocite[8]{Liu.2009b}. A more detailed description was found also in \cite{Iwata.2011}, where physical interaction is explicit identified as a driver to enhance emotional
engagement.\autocite[cf.][8]{Iwata.2011} Additionally this benefit was found in \cite{Li.2011} and \cite{Hou.2013}.\autocite[cf.][322]{Li.2011}\mulcit\autocite[cf.][448]{Hou.2013}
\heading{Increased Attention}
The benefit 'Increased Attention' is about the amount of attention the user pays to the technology and with this to the teaching and learning content. It is once mentioned literally by \cite{VateULan.2012}.\autocite[cf.][894]{VateULan.2012} In the other two cases, we interpreted the quotations "felt it interesting [...] using the AR-guide system"\autocite[194]{Chen.2008} and 
"teachers noted that the smartphones [the AR-System] promoted interaction with the pond (of which the pupils should learn something about) and classmates"\autocite[552]{Kamarainen.2013} as proof for 
increased attention. As said at the beginning of this chapter, such cases could be interpreted in other ways.
\heading{Increased Concentration}
The difference between the benefits 'increased attention' and 'increased concentration' is just the fact, that we found it literally in the literature. We included it
without any further interpretation or consideration to possible overlapping in the meaning of the terms.
'Increased Concentration' is about the amount of concentration while using \AR \appsns. This benefit was also found for three times. Similar to the detailed description of \cite{Iwata.2011} for increased motivation through \AR \appns, "physical interaction
induced deeper concentration [...]"\autocite[9]{Iwata.2011}, too. \cite{Yen.2013} as well as \cite{Ibanez.2014} perceive an "higher [...] degree of concentration"\autocite[173]{Yen.2013} respectively a 
"higher level of concentration"\autocite[11]{Ibanez.2014}. 
\heading{Increased Satisfaction}
'Increased Satisfaction' means that users experience higher satisfaction regarding the learning process or that users were more satisfied regarding what they have learned after learning with the \AR \app than with the 
conventional method. As an example of more satisfaction regarding the learning process, students have more fun running through a library an solve some tasks directed by an \AR \app than by a librarian\autocite[649]{Chen.2012}.
\cite{MartinGutierrez.2013} says, that "the students were quite satisfied with the [AR-]tools used to learn"\autocite[6]{MartinGutierrez.2013}. A reverse statement therefor is that the frustration level is higher using the 
manual way\autocite[cf.][448]{Hou.2013}. Additionally this benefit is mentioned by \cite{Ibanez.2014} and \cite{Redondo.2013}, so in total five times.

% 
\subsubsection{Teaching Concepts}
During our analysis we observed that two different teaching concepts were supported by AR applications. We clustered these concepts as 'Student Centered Learning' and 'Collaborative Learning' which we explain in the following.
\heading{Increased Student Centered Learning}
Student centered learning is a teaching concept where conventional lectures are replaced by new active and self-paced learning programs. In student centered learning approaches, students are more self-responsible for their own progress in education and educators act as faciliators, who enable the students to learn independently and individualized.\\
Three studies report that AR enabled an increased student centered learning approach in the regarded learning environment. \cite{VateULan.2012} recognizes that the regarded AR application enabled "functionality depended on [...] students’ learning capability" \autocite [894]{VateULan.2012}. Similarly, \cite{Kamarainen.2013} report that "these technologies provide ways of individualizing instruction in a group setting".\autocite[554]{Kamarainen.2013} In addition, \cite{Kamarainen.2013} state that "the technology supported independence" which "freed the teacher to act as a faciliator".\autocite[554]{Kamarainen.2013} Furthermore, \cite{Liu.2009b} report that AR "improves the ability to explore and absorb new knowledge and solve problems" \autocite[173]{Liu.2009b} which indicates that AR can support student-centered learning environments as students are enabled to explore knowledge and solve problems autonomously. These studies show, that AR can support a student centered learning approach by providing educators with new possibilites to individualize their lessons to students' capability and by enabling students to learn more independently from educators.
\heading{Improved Collaborative Learning}
Three studies report that the regarded AR application improved collaborative learning, meaning that AR enabled new ways of communication and cooperation. \cite{Wang.2012} regard their AR application as "effective environment for conducting collaborative inquiry learning activities". \autocite[57]{Wang.2012} Other authors join the observation of improved collaborative learning as they highlight "the opportunity for collaborative communication and problem-solving among students that arose from the augmented reality experience" \autocite[552]{Kamarainen.2013} and the "facilitation effects of AR technology on collaborative learning effectiveness".\autocite[322]{Li.2011}
% 
\subsubsection{Presentation}
All benefits in the group 'Presentation' are related to the way in which content which should be learned or taught or objects which should support the learning process are visually presented to the user.  
\heading{Increased Details}
The benefit 'Increased Details' is mentioned once. In the context of urban design education the tested \AR "has more detailing particular in the texture of models"\autocite[17]{Chen.2008} than using wood block models of objects for urban 
design learning, as it is the case in the traditional learning method.
\heading{Increased Information Accessibility}
It is reported twice, that \AR \apps improves and eases the access to information regarding the teaching and learning content. In the context of an assembly task guided by an \AR \app instead of an conventional 
assembly manual, \cite{Hou.2013} reports that "[...] AR eases information retrieval by integrating the task of searching information and the task of the actual assembly" \autocite[447]{Hou.2013}. Also \cite{Iwata.2011}
mentions, that "superimposed information was nicely integrated and did not interfere with the learning process"\autocite[112]{Iwata.2011} while learning a traditional Chinese board game.
\heading{Increased Interactivity}
The benefit 'Increased Interactivity' could be seen as a precondition for other benefits presented in this paper, which it is, as you can see in the following and as noted in benefit 'Increased Motivation'. Anyway, increased interactivity through the application of \AR is a 
fact which is not recognised by application of the corresponding conventional method\autocite[cf.][113]{Dunser.2012}\mulcit\autocite[cf.][11]{Ibanez.2014} and therefore specified as a single benefit.
\cite{Dunser.2012} states, that "tangible interaction using tools such as the magnet paddle and augmented nail with labeled poles [in the context of teaching physics] allow for a learning experience that combines real world objects with virtual content. 
Together this can contribute to a deeper understanding. Interactions in AR engage learners with the content, and allow for knowledge to be acquired through their own manipulation of content [...], 
as supported by constructivist learning theory [...]."

% 
\subsubsection{Learning Type}
This subsection deals with benefits we clustered as 'Learning Type'. This group contains benefits which were linked to a specific type of learning, for instance creativity or a more theoretical learning approach like language education. \\
Therefore this group contains two sub-items: improved learning curve and increased creativity. While an improved learning curve is observable on skills based learning, such as spatial skills, or on fields which require a logical understanding, such as languages, increased creativity can be observed on less theoretical grounded areas, such as problem solving or arts.

\heading{Improved Learning Curve}
An improved learning curve, meaning that students learn faster and easier with \AR \apps compared to non-\AR \appsns, is the most often mentioned benefit of \ARns. A total of 26.87\% of all benefits mentioned were related to an improved learning curve. \\
\cite{Liu.2009} reports that "tests taken by the experimental group [the \AR \app users] in all the learning activities were significantly better than those of the control group [the traditionally learning users]".\autocite[525]{Liu.2009} Similarly, \cite{Chang.2014} states, that "[t]he AR-guided group had better learning effectiveness (as evidenced by their posttest scores), and it was found that most visitors believed the AR guide made it easier to digest information than the audio guide due to the extra visual commentary that is provided"\autocite[193]{Chang.2014} as well as "[t]he learning performance of the AR-guided group was thus superior to
that of the other two groups"\autocite[190]{Chang.2014}. More authors join this observation like \cite{Kamarainen.2013} ("[w]e witnessed significant learning gains"\autocite[550]{Kamarainen.2013}), \cite{Ibanez.2014} ("it was found that students who used the AR application performed significantly better on knowledge"\autocite[12]{Ibanez.2014}), \cite{Li.2011}, \cite{MartinGutierrez.2011}, \cite{Redondo.2013}, \cite{Liu.2009b} ("achieved significantly more learning improvement"\autocite[173]{Liu.2009b}), \cite{Zhang.2014}, \cite{Yeo.2011}, \cite{Hou.2013} ("[AR] shortens the learning curve"\autocite[450]{Hou.2013}, "[the] learning curve of trainees significantly improved"\autocite[451]{Hou.2013}), \cite{Wilson.2013} and \cite{Anderson.2013} ("learning [results] increased by more than a factor of 2"\autocite[318]{Anderson.2013}). 
\heading{Increased Creativity}
Increased creativity was mentioned three times (which makes 4,48\% of all reported benefits). For instance, \cite{Liu.2009b} found that "it [AR] also improves student creativity and the ability to explore and absorb new knowledge and solve problems"\autocite[173]{Liu.2009b}. \cite{VateULan.2012} reports, that the "AR 3D pop-up book has highlighted many benefits that include: [...] integration of a variety of learning skills such as [...] and creativity [...]"\autocite[894]{VateULan.2012}. Also, \cite{Chang.2014} observes, that "[o]verall the visitors using the mobile AR-guide system during painting appreciation activities felt that it was an interesting, innovative, creative, and entertaining guide device"\autocite[194]{Chang.2014}. To increase the interpretability of the impact of \AR \apps on creativity, more studies are needed.
% 
\subsubsection{Content Understanding}
The subsection 'Content Understanding' deals with benefits related to the understanding of the learning content by the user and with this the ability to keep the content in memory.
\heading{Improved Development of Spacial Abilities}
The benefit of 'Improved Development of Spacial Abilities' is mentioned four times. \cite{Dunser.2012} for instance says that their "results support the hypothesis, and suggest that Augmented Reality has some potential to be effective 
in aiding the learning of 3D concepts"\autocite[112]{Dunser.2012}. Literally and more detailed, the benefit was found in \cite{MartinGutierrez.2011} as he says that "the training of spatial ability based on 
Graphic Engineering contents and AR technology improves spatial abilities for those who perform them and consequently lower the numbers of students who drop out of the subject"\autocite[5]{MartinGutierrez.2011}.
Additionally this benefit is mentioned by \cite{MartinGutierrez.2013} and \cite{Chen.2008}.\autocite[cf.][4]{MartinGutierrez.2013}\mulcit\autocite[cf.][5]{Chen.2008}
\heading{Improved Memory}
'Improved Memory' refers to the retention of what was learned during the application of the \ARns-method. \cite{Hou.2013} states that "trainees with AR training could remember or recollect more assembly 
clues that were memorized in the former training task than those trained in the manual"\autocite[450]{Hou.2013}. Furthermore it is not only about the mere memory, but also about how vivid the memory is. As \cite{Chang.2014}
says "It [the \AR \app] facilitates the development of art appreciation by imprinting the knowledge of paintings on visitor's memories, supporting the coupling between the visitors, the guide system, and the artwork (Klopfer \& Squire, 2008) by using 
AR technology, and helping visitors keep their memories of the artwork vivid"\autocite[193]{Chang.2014}. Also \cite{Macchiarella.2005} says that \AR "lead[s] to an increased ability to retain long term memories"\autocite[4]{Macchiarella.2005}.
This benefit is mentioned three times in total.

% 
\subsubsection{Reduced Cost}
\cite{Leblanc.2010} and \cite{MartinGutierrez.2011} reported reduced costs in \ARns-scenarios compared to traditional learning in long term. \cite{Chen.2012} highlights especially the low cost in executing manpower and moderate costs for designing and renewing of courses.\autocite[cf.][640]{Chen.2012} \cite{Andujar.2011} join in this point, especially for virtual laboratories.\autocite[cf.][492]{Andujar.2011} \cite{Andujar.2011} add that \ARns-\apps not only reduce direct costs, such as needed materials, but also time for preparing classes. While, at least at the time of this review, \ARns-technology is accompanied with high aquisition cost, this investment will most likely be paid off in the long term. \cite{Leblanc.2010} report, that the one time acquisition cost were high (25.000 US-Dollar)\autocite[253]{Leblanc.2010}, but the cost per class could be lowered by 93,34\% (from 3.000 US-Dollar to 200 US-Dollar)\autocite[253]{Leblanc.2010} which will lead to an overall cost reduction.