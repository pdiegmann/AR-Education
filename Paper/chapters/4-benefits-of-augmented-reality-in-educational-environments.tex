\section{Benefits of Augmented Reality in Educational Environments}
\subsection{Benefit Categorization}
\label{subsec:Benefits}

% 
\subsubsection{State of Mind}
\heading{Increased Motivation}
\heading{Increased Attention}
\heading{Increased Concentration}
\heading{Increased Satisfaction}

% 
\subsubsection{Teaching Concepts}
\heading{Increased Student Centered Learning}
\heading{Improved Collaborative Learning}

% 
\subsubsection{Presentation}
\heading{Increased Details}
\heading{Increased Information Accessibility}
\heading{Increased Interactivity}

% 
\subsubsection{Learning Type}
This subsection deals with benefits we clustered as 'Learning Type'. This group contains benefits which could be linked to a specific type of learning, for instance creativity or a more theoretical learning approach like language education. \\
Therefor this group contains two subitems: improved learning curve and increased creativity. While an improved learning curve is observable on skills based learning, such as spatial skills, or on fields which require a logical understanding, such as languages, increased creativity can be observed on less theoretical grounded areas, such as problem solving or arts.

\heading{Improved Learning Curve}
An improved learning curve, meaning that students learn faster and easier with \AR \apps compared to non-\AR \appsns, is the most often mentioned benefit of \ARns. A total of 26.87\% of all benefits mentioned were related to an improved learning curve. \\
\cite{Liu.2009} reports that "tests taken by the experimental group [the \AR \app users] in all the learning activities were significantly better than those of the control group [the traditionally learning users]".\autocite[525]{Liu.2009} Similarly, \cite{Chang.2014} states, that "[t]he AR-guided group had better learning effectiveness (as evidenced by their posttest scores), and it was found that most visitors believed the AR guide made it easier to digest information than the audio guide due to the extra visual commentary that is provided"\autocite[193]{Chang.2014} as well as "[t]he learning performance of the AR-guided group was thus superior to
that of the other two groups"\autocite[190]{Chang.2014}. More authors join this observation like \cite{Kamarainen.2013} ("[w]e witnessed significant learning gains"\autocite[550]{Kamarainen.2013}), \cite{Ibanez.2014} ("it was found that students who used the AR application performed significantly better on knowledge than those who were taught using the web- based application"\autocite[12]{Ibanez.2014}), \cite{Li.2011}, \cite{MartinGutierrez.2011}, \cite{Redondo.2013}, \cite{Liu.2009b} ("achieved significantly more learning improvement"\autocite[173]{Liu.2009b}), \cite{Zhang.2014}, \cite{Yeo.2011}, \cite{Hou.2013} ("shortens the learning curve"\autocite[450]{Hou.2013}, "learning curve of trainees significantly improved"\autocite[451]{Hou.2013}), \cite{Wilson.2013} and \cite{Anderson.2013} ("learning increased by more than a factor of 2"\autocite[318]{Anderson.2013}). 
\heading{Increased Creativity}
Increased creativity was mentioned three times (which makes 4,48\% of all reported benefits). For instance, \cite{Liu.2009b} found that "it also improves student creativity and the ability to explore and absorb new knowledge and solve problems"\autocite[173]{Liu.2009b}. \cite{VateULan.2012} reports, that the "AR 3D pop-up book has highlighted many benefits that include: [...] integration of a variety of learning skills such as [...] and creativity [...]"\autocite[894]{VateULan.2012}. Also, \cite{Chang.2014} observes, that "[o]verall the visitors using the mobile AR-guide system during painting appreciation activities felt that it was an interesting, innovative, creative, and entertaining guide device"\autocite[194]{Chang.2014}. To increase the interpretability of the impact of \AR \apps on creativity, more studies are needed.
% 
\subsubsection{Content Understanding}
\heading{Improved Development of Spacial Abilities}
\heading{Improved Memory}

% 
\subsubsection{Reduced Cost}
\cite{Leblanc.2010} and \cite{MartinGutierrez.2011} reported reduced costs in \ARns-scenarios compared to traditional learning in long term. \cite{Chen.2012} highlights especially the low cost in executing manpower and moderate costs for designing and renewing of courses.\autocite[cf.][640]{Chen.2012} \cite{Andujar.2011} joins in this point, especially for virtual laboratories.\autocite[cf.][492]{Andujar.2011} \cite{Andujar.2011} add that \ARns-\apps not only reduce direct costs, such as needed materials, but also time for preparing classes. While, at least at the time of this review, \ARns-technology is accompanied with high aquisition cost, but this investment most likely paid of in the long term. \cite{Leblanc.2010} report, that the one time acquisition cost were quite high (25.000 US-Dollar)\autocite[253]{Leblanc.2010}, but the cost per class could be lowered by 93,34\% (from 3.000 US-Dollar to 200 US-Dollar)\autocite[253]{Leblanc.2010} which will lead to cost reduction.