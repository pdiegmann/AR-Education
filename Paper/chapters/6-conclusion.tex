\section{Conclusion}
Each \AR \app is unique regarding its implementation and especially its purpose on how to support which subject. Hence it is difficult to generalize the benefits, even if they are mentioned literally in the articles.
By doing so, there should be paid special attention on how the author uses the terms of the benefit and how he understand them to prevent misunderstandings and bad, unclean results. Further the application has to be a final 
and well implemented version and no kind of test version to derive valid benefits of \AR from its application and empirical test. The usability of the \AR application for instance should be at its maximum to 
benefit from its application and potential benefits. \\
All in all and apart from that, we found 14 different, frequently mentioned benefits of \AR in our source literature out of which two benefits ('Improved Learning Curve' and 'Increased Motivation') account for over 20% 
of all benefits mentioned. Hence, on these benefits should be a focus in future work assessing \AR \apps in educational environments under consideration of what was noted in the discussion chapter.


%Evtl Zusätzlich in limitations (oder Conclusion): Jede AR application ist einzigartig, daher sind benefits nur sehr schwer generalisierbar. Eine Applikation muss auf jeden Fall vernünftig umgesetzt sein (z.B. ausreichendes Maß an usability) um von den potentiellen Vorteilen zu profitieren.
%(SVEN)------ sollte auf jendefall in limitations finde ich ----->Evtl Zusätzlich in limitations (oder conclusion): Wir haben keine "Special Learner" betrachtet. Gerade für diese Gruppe ist aber AR besonders interessant! (Evtl. Zitat) Erklären warum das notwendig war und dass man diese Gruppen gesondert betrachten sollte, da hier erhebliches Potential für education besteht!