\section{Conclusion}
\label{sec:Conclusion}
Finally, we want to stress, that each \AR \app is in it's own way unique and therefor it is not always easy to generalise. Each \app has to be implemented thoroughly to prevent drawbacks in user interaction or system failures in order to profit from benefits.

Additionally, we imagine that 'special learners', e.g. handicapped people, can derive different as well as additional benefits out of \AR \apps for learning purposes due to their special requirements to learning methods and the special characteristics of \ARns. The exploration of these benefits could be an objective for future research in the field of \AR \apps in educational environments.
We found 14 different benefits of \AR in our source literature out of which two benefits ('Improved Learning Curve' and 'Increased Motivation') each accounts for over 20\% of all benefits mentioned. Hence, other benefits with much lower representation could be focused on in future work assessing \AR \apps in educational environments. Similarly, future research should focus on single directions of the Five Directions.\\
To draw a conclusion, \AR is eligible to be used in educational environments and we found many \apps which successfully used \AR to improve learning, i.e. in language education, training of mechanical skills and spatial abilities training. Nevertheless, \AR is no magic bullet to educational environments.


%Evtl Zusätzlich in limitations (oder Conclusion): Jede AR application ist einzigartig, daher sind benefits nur sehr schwer generalisierbar. Eine Applikation muss auf jeden Fall vernünftig umgesetzt sein (z.B. ausreichendes Maß an usability) um von den potentiellen Vorteilen zu profitieren.
%(SVEN)------ sollte auf jendefall in limitations finde ich ----->Evtl Zusätzlich in limitations (oder conclusion): Wir haben keine "Special Learner" betrachtet. Gerade für diese Gruppe ist aber AR besonders interessant! (Evtl. Zitat) Erklären warum das notwendig war und dass man diese Gruppen gesondert betrachten sollte, da hier erhebliches Potential für education besteht!

% alternative object raus
% and apart from that raus

%By doing so, there should be paid special attention on how the author uses the terms of the benefit and how the author understands them to prevent misunderstandings and bad, unclean results. Further the application should be mature and no prototype to derive benefits of \AR from its application and empirical test. The usability of the \AR application for instance should be at its maximum to benefit from its application and potential benefits. \\