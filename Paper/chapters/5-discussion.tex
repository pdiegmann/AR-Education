\section{Discussion}
\label{sec:Discussion}
%Beginn mit Vergleich zu Radu: Ähnlichkeiten: Spatial abilities, long term memory, collaboration, motivation. Seine Einzelbenefits (Content Understanding, language association, physical task performance) wird bei uns als improved learning curve zusammengefasst und ergibt sich bei uns aus der Art der Applikation und dem benefit learning curve. Z.B. direction "skills training" mit benefit learning curve = improved physical task performance. Den Benefit spatial abilities haben wir hingegen als eigenen Benefit gewählt weil durch den Einsatz von AR ein neues Level von Spatial abilities erreichen können und nicht nur die learning curve improved ist. (Evtl gutes Zitat)
%Unterschiede: Motivation haben wir aufgeteilt in 2 Unterkategorien. Hier nochmal begründen warum wir das gemacht haben und dass wir das durchaus für Sinnvoll erachten da Unterscheidungen zu treffen. Attention hat er als Nachteil gecoded. Da also durchaus ein großer Unterschied. Evtl der Hinweis darauf, dass beides der Fall sein kann, aber bei gut designter Applikation eher Vorteil (Gibt ein sehr gutes Zitat mit diesen ponds wo die autoren eine sinkende Attention befürchtet haben aber genau das Gegenteil war der Fall!)
% Benefits die wir zusätzlich haben: Reduced Costs, student-centered learning, creativity und alle presentation benefits.

% Guter Übergang um auf creativity und student-centered learning. Hier nochmal darauf eingehen, dass wir nicht damit gerechnet hätten, dass auch die creativity so gut unterstützt wird. (AR als lineares Tool (Phil hatte da was gutes gesagt? :D). Creativity zeigt, dass AR wirklich ein Tool ist, was in enorm vielen educational Bereichen eingesetzt werden kann. Student-centered learning ist besonders interessant, da gerade der konventionelle Frontunterricht in viele Fällen nicht mehr zeitgemäß ist (Hier wäre ein gutes Zitat wichtig) und durch neue Lehrformen wie das student-centered learning ersetzt wird. AR kann gerade in diesem Bereich helfen diese neuen Lehrformen voran zu bringen und die Lehre zu verbessern.
% Besonders Geeignet scheint hierbei auch der discovery-based learning approach zu sein, der offensichtlich sehr gut von AR unterstütz wird. (Nochmal kurz auf die vielen Benefits eingehen) Evtl auch auf Verbindung discovery-based und student-centered learning eingehen, da discovery-based eine Ausprägung von student-centered sein kann? --> Lernen der Zukunft?


Our study is limited by a number of factors. Firstly, some of the regarded empirical studies are only informal investigations with a low number of participants. The significance of the ascertained benefits of AR applications may be unclear in these cases. In addition, for some of the regarded directions we did not find enough articles in order to make a point about the diversity of benefits in comparison to other directions. However, AR is one of the most emerging technologies in education and the fact that 15 out of 25 articles we regarded were published in 2012 or later shows that these limitations can be overcome in the future when more empirical evaluations of AR applications in educational environments will be published. Once enough articles have been published we would suggest to investigate every direction of AR in education separately with a decent amount of regarded articles in order to find out more about the diversity of benefits between directions. \\
Another factor which limits our study is revealed by the inter-code reliability of 0.64 regarding the classification of articles to a certain direction of AR. We think that this rather low value can be explained by the circumstance that some articles can not precisely be classified to a single direction, e.g. a discovery-based learning application which uses game elements. In addition, the definitions by \cite{Yuen.2011} leave some room for interpretation which we tried to reduce during our systematic literature review.

% Evtl Zusätzlich in limitations: Wir haben nur benefits betrachtet. Diese können natürlich auch von Nachteilen aufgehoben werden. Radu nennt potentielle Nachteile in seinem Paper.
%Evtl Zusätzlich in limitations (oder conclusion): Wir haben keine "Special Learner" betrachtet. Gerade für diese Gruppe ist aber AR besonders interessant! (Evtl. Zitat) Erklären warum das notwendig war und dass man diese Gruppen gesondert betrachten sollte, da hier erhebliches Potential für education besteht!
%Evtl Zusätzlich in limitations (oder Conclusion): Jede AR application ist einzigartig, daher sind benefits nur sehr schwer generalisierbar. Eine Applikation muss auf jeden Fall vernünftig umgesetzt sein (z.B. ausreichendes Maß an usability) um von den potentiellen Vorteilen zu profitieren.