\section{Discussion}
\label{sec:Discussion}
Hier blabla discussion \\

Our study is limited by a number of factors. Firstly, some of the regarded empirical studies are only informal investigations with a low number of participants. The significance of the ascertained benefits of AR applications may be unclear in these cases. In addition, for some of the regarded directions we did not find enough articles in order to make a point about the diversity of benefits in comparison to other directions. However, AR is one of the most emerging technologies in education and the fact that 15 out of 25 articles we regarded were published in 2012 or later shows that these limitations can be overcome in the future when more empirical evaluations of AR applications in educational environments will be published. Once enough articles have been published we would suggest to investigate every direction of AR in education separately with a decent amount of regarded articles in order to find out more about the diversity of benefits between directions. Another factor which limits our study is revealed by the inter-code reliability of 0.64 regarding the classification of articles to a certain direction of AR. We think that this rather low value can be explained by the circumstance that some articles can not precisely be classified to a single direction, e.g. a discovery-based learning application which uses game elements. In addition, the definitions by \cite{Yuen.2011} leave some room for interpretation which we tried to reduce during our systematic literature review.