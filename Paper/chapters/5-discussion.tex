\section{Discussion}
\label{sec:Discussion}
%Beginn mit Vergleich zu Radu: Ähnlichkeiten: Spatial abilities, long term memory, collaboration, motivation. Seine Einzelbenefits (Content Understanding, language association, physical task performance) wird bei uns als improved learning curve zusammengefasst und ergibt sich bei uns aus der Art der Applikation und dem benefit learning curve. Z.B. direction "skills training" mit benefit learning curve = improved physical task performance. Den Benefit spatial abilities haben wir hingegen als eigenen Benefit gewählt weil durch den Einsatz von AR ein neues Level von Spatial abilities erreichen können und nicht nur die learning curve improved ist. (Evtl gutes Zitat)
In comparison to \cite{Radu.2014}, our study has some similarities as well as some distinctions. \cite{Radu.2014} mentions "spatial abilities", "long term memory, collaboration" and "motivation". We found these benefits as well and therefor inherited them. But in contrast to \cite{Radu.2014}, we condensed "content understanding", "language association" and "physical task performance" into "improved learning curve". Depending on the direction of the \appns, we are able to disaggregate our condensed "improved learning curve" benefit into a more detailed benefit, i.e. a \ST \app with an improved learning curve is equal to "physical task performance". We have to mention, that we have chosen to define "Development of Spatial Abilities" as another benefit and even in another group, as some \apps lead to a new level of spatial abilities which might not have been achieved without \AR or is at least extraordinary improvements in spatial abilities. \cite{MartinGutierrez.2013} states, that "[...] the students have a probability of over 95\% of improving their levels of spatial ability when performing the proposed training. Besides this, results show there is no improvement in control group levels"\autocite[4]{MartinGutierrez.2013} which indicates that spatial abilities were improved far more than usual. \\
% passt das? "[t]he training of spatial ability based on Graphic Engineering contents and AR technology improves spatial abilities [...] lower[s] the numbers of students who drop out of the subject."\autocite[5]{MartinGutierrez.2011}

A larger difference compared to \cite{Radu.2014} is, that we segmented attention into two subcategories, namely concentration and attention. While \cite{Radu.2014} states that \AR \apps might fail to improve student's attention or lead to an unintended focus on the technology itself and not the topic\autocite[cf.][314]{Radu.2014}, we found articles that state the opposite. \cite{Kamarainen.2013} states, that "[t]he teachers stated that they began this project with skepticism about whether the technology would overwhelm the experience, holding the students’ attention at the expense of their noticing the real environment. However, teachers and investigators found the opposite to be true. Students were captivated when a squirrel dropped a seed from a tree near the path and nearly hit a classmate; they called out excitedly when they observed a frog near the shore."\autocite[554]{Kamarainen.2013}, and therefor we think, that the drawback mentioned by \cite{Radu.2014} might be related to system design. Furthermore our segmentation into attention and concentration is based on the findings by \cite{Kamarainen.2013}. Attention relates to only an increased awareness of the situation and a focus on the broader environment, while concentration refers to an increased awareness of the topic or subject and an high level of cognitive activity.

% Unterschiede: Motivation 
%
% !!! MEINST DU EVTL. CONCETRATION / ATTENTION? !!! 
% 
% haben wir aufgeteilt in 2 Unterkategorien. Hier nochmal begründen warum wir das gemacht haben und dass wir das durchaus für Sinnvoll erachten da Unterscheidungen zu treffen. Attention hat er als Nachteil gecoded. Da also durchaus ein großer Unterschied. Evtl der Hinweis darauf, dass beides der Fall sein kann, aber bei gut designter Applikation eher Vorteil (Gibt ein sehr gutes Zitat mit diesen ponds wo die autoren eine sinkende Attention befürchtet haben aber genau das Gegenteil war der Fall!)

Some benefits we found were not mentioned by \cite{Radu.2014}, namely reduced costs, student-centered learning, creativity as well as all presentation-related benefits, like increased details, easy accessible information and interactivity.\\
% Benefits die wir zusätzlich haben: Reduced Costs, student-centered learning, creativity und alle presentation benefits.

Regarding increased creativity as one benefit of \AR \appsns, we would not have thought of creativity as one benefit in advance. On the contrary, we would have assumed, that a linear learning tool as \AR \appsns, which are only able to display information that someone added by hand and interact in ways which are predefined. Our findings conversely show that a linear learning tool as \AR \apps is able to support creative, non-linear learning. This finding also stresses, that \AR is a very flexible tool, which can be used in many educational environments and settings and for very different purposes - if it is applied thoroughly. % TODO: Neue Lehrformen und so
\\
% Guter Übergang um auf creativity und student-centered learning. Hier nochmal darauf eingehen, dass wir nicht damit gerechnet hätten, dass auch die creativity so gut unterstützt wird. (AR als lineares Tool (Phil hatte da was gutes gesagt? :D). Creativity zeigt, dass AR wirklich ein Tool ist, was in enorm vielen educational Bereichen eingesetzt werden kann. Student-centered learning ist besonders interessant, da gerade der konventionelle Frontunterricht in viele Fällen nicht mehr zeitgemäß ist (Hier wäre ein gutes Zitat wichtig) und durch neue Lehrformen wie das student-centered learning ersetzt wird. AR kann gerade in diesem Bereich helfen diese neuen Lehrformen voran zu bringen und die Lehre zu verbessern.

\DBL seems to be a very promising \AR direction. As outlined in \ref{subsubsec:DiscoveryBasedLearning} it has benefits ranging from increased motivation, improved learning curve to reduced costs and supports student-centered learning, as supporting a \DBLns-approach, the student is the center of the learning process and the learning process is adjusted to the student's needs and preferences. We could imagine this to be the way students learn in future. \\
% Besonders Geeignet scheint hierbei auch der discovery-based learning approach zu sein, der offensichtlich sehr gut von AR unterstütz wird. (Nochmal kurz auf die vielen Benefits eingehen) Evtl auch auf Verbindung discovery-based und student-centered learning eingehen, da discovery-based eine Ausprägung von student-centered sein kann? --> Lernen der Zukunft?
\\
Our study is limited by a number of factors. Firstly, some of the regarded empirical studies are only informal investigations with a low number of participants. The significance of the ascertained benefits of AR applications may be unclear in these cases. In addition, for some of the regarded directions we did not find enough articles in order to make a point about the diversity of benefits in comparison to other directions. However, AR is one of the most emerging technologies in education and the fact that 15 out of 25 articles we regarded were published in 2012 or later shows that these limitations can be overcome in the future when more empirical evaluations of AR applications in educational environments will be published. Once enough articles have been published we would suggest to investigate every direction of AR in education separately with a decent amount of regarded articles in order to find out more about the diversity of benefits between directions. \\

Another factor which limits our study is revealed by the inter-code reliability of 0.64 regarding the classification of articles to a certain direction of AR. We think that this rather low value can be explained by the circumstance that some articles can not precisely be classified to a single direction, e.g. a discovery-based learning application which uses game elements. In addition, the definitions by \cite{Yuen.2011} leave some room for interpretation which we tried to reduce during our systematic literature review.
\\

While \cite{Radu.2014} states also (potential) negative aspects of \AR in educational environments, we focused on benefits, although negative aspects might offset benefits. \\
% Evtl Zusätzlich in limitations: Wir haben nur benefits betrachtet. Diese können natürlich auch von Nachteilen aufgehoben werden. Radu nennt potentielle Nachteile in seinem Paper.

Another aspect we left out are "special learners": while handicapped people have (sometimes) special requirements, we focused on more general aspects of \AR in educational environments. \\
%Evtl Zusätzlich in limitations (oder conclusion): Wir haben keine "Special Learner" betrachtet. Gerade für diese Gruppe ist aber AR besonders interessant! (Evtl. Zitat) 
% 
% Vielleicht diesen Teil eher in der Conclusion, aka future research?
% 
%Erklären warum das notwendig war und dass man diese Gruppen gesondert betrachten sollte, da hier erhebliches Potential für education besteht!

% Siehe Conclusion: Finally, we want to stress, that each \AR \app is in it's own way unique and there for it is not always easy to generalise. Each \app has to be implemented thoroughly to prevent drawbacks in user interaction or system failures in order to profit from benefits.
%Evtl Zusätzlich in limitations (oder Conclusion): Jede AR application ist einzigartig, daher sind benefits nur sehr schwer generalisierbar. Eine Applikation muss auf jeden Fall vernünftig umgesetzt sein (z.B. ausreichendes Maß an usability) um von den potentiellen Vorteilen zu profitieren.