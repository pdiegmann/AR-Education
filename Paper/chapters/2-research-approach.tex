\section{Research Approach}
We applied a two-step research approach, whereby we first conducted a systematic literature review to identify relevant publications before analysing the identified publications for the coding of benefits and directions. After coding, we grouped all found benefits.

\subsection{Systematic Literature Review}
For the identification of papers addressing Augmented Reality in educational environments, we applied a systematic online literature database search. We included databases which were specialised on more information systems centered papers, namely Institute of Electrical and Electronic Engineers (IEEE) Xplore Digital Library, ProQuest (ABI / INFORM), Association for Information Systems Electronic Library (AISel) and Association for Computing Machinery (ACM) Digital Library, as well as more general databases, namely EBSCO Host and ScienceDirect.\\
To find relevant papers, we searched within all databases with on the following attributes: title, abstract and author supplied keywords. Within these keywords we had three mandatory groups of keywords. Every article had to include the keyword "Augmented Reality". Additionally, every article needed to have at least one synonym for education and benefits. Namely we searched for "Educat*", "Learn*", "Teach*", "College" or "School" as synonyms for education and "Benefi*", "Advan*", "Improv*", "Enhanc*", "Driver*" or "Value*" as synonyms for benefits. To deal with the limitations of some databases, we had to split our query and conduct multiple queries on the database and merge them together by hand.\\
This database query lead to a total of 600 results. Those results then were checked against our include- and exclude-criteria and were coded into one of the five directions. This process was performed by ourselves and each article was read by two of the authors.

\subsection{Data Analysis}